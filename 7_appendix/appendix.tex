
% \appendix

%%%%%%%%%%%%%%%%%%%%%%%%%%%%%%%%%%%%%%%%%%%%%%%%%%%%%%%%%%%%%%%%%%%%%%%%%%%%%%%%
 
\chapter{Numerical integration of the Fokker-Planck equation using the Crank-Nicolson method} \label{app_sec:numerical_integration_with_crank_nicolson}

%%%%%%%%%%%%%%%%%%%%%%%%%%%%%%%%%%%%%%%%%%%%%%%%%%%%%%%%%%%%%%%%%%%%%%%%%%%%%%%%

For executing the numerical integration of a FP equation in the form of Eq.~\eqref{eq:fp}, we used the Crank-Nicolson integration scheme~\cite{crank1947practical}, which is a finite difference method, second-order and implicit in time. It can be shown that this scheme is unconditionally stable for many differential equations~\cite{thomas2013numerical}.

To obtain valid numerical results of the integration of the FP equation with Nekhoroshev-like diffusion coefficient like in Eq.~\eqref{eq:diffusion}, and obtain a consistent evaluation of the outgoing current in different scenarios, we have to properly evaluate the stiffness of the problem and adapt consequently the fineness in both time and space discretisations. Moreover, concerning the simulation of an instantaneous change in position of the absorbing boundary condition, a rigorous protocol must be established, especially when considering the inward displacements of the boundary condition, for which some additional precautions must be taken.

The outgoing current, defined in Eq.~\eqref{eq:outgoing_current_definition}, is obtained by computing directly, in between each integration step, the numerical derivative of $\rho$ at the absorbing boundary position. 

A Nekhoroshev-like diffusion coefficient has the main characteristic of varying by various orders of magnitude over the accessible range of the the action variable, meaning that if we want to simulate the entirety of a diffusive phenomenon, we must take into consideration such a wide range of values in the integration process. This becomes mostly critical when the process to be simulated is the recovery current that occurs after a variation of the position of the boundary condition, like the ones described in Section~\ref{sec:moving_the_absorbing_barrier}.

The recovery current is mainly dependent on variations in the equilibrium distribution that are various orders of magnitude lower, in absolute value, than the core part of the distribution (refer to Fig.~\ref{fig:3} and~\ref{fig:5}). It is therefore necessary to choose a time and space discretisation fine enough to obtain numerical estimates that are not seriously affected by the integration error. To do that, we performed a convergence test for a single recovery current in every scenario we wanted to analyse. In such a convergence test, we increased gradually the fineness of the discretisations, until we measured a relative difference between numerical results not higher than $1\%$.

When it comes instead to reproduce the instantaneous change of the position of the absorbing boundary condition in the integration scheme, we perform a re-sampling of the distribution $\rho$ at the time of the boundary change, while keeping the same fineness for the spatial discretisation. In an outward movement, however, this process is straightforward, as there is no artificial change to the existing distribution $\rho$ to be taken into account, and we just add an empty region with no singular points. Instead, for the case of an inward movement, we do have to perform a cut inside the $\rho$ distribution, corresponding to the movement performed by the absorbing boundary. Such a cut generates an inconsistency between the non-zero value of $\rho$ at the new position of the boundary condition and the zero condition imposed by the absorbing boundary condition. This inconsistency leads to a divergence in the analytical definition of the outgoing current and undefined behaviours in the numerical integration. Therefore, we apply to the cut distribution a sharp damping, right next to the newly positioned absorbing boundary condition, generated by a logistic function $f(I)$ defined as
\begin{equation}
    f(I) = \frac{1}{1 + e^{\frac{I-I_\text{a}+\ell}{\ell}}} \, , 
    \label{eq:logistic_damping}   
\end{equation}
where $\ell$ is the extent of the range of action values where the damping occurs, and is taken equal to two twice the fineness of the spatial discretisation, and $I_\text{a}$ is the position of the absorbing boundary condition after the inward movement. In this way, $\rho_d(I) = \rho(I) f(I)$ represents a distribution that is smooth enough to avoid instabilities in the numerical integration. The sharpness of this damping is directly proportional to the fineness of the spatial sampling, and its effects are included in the convergence tests.

%%%%%%%%%%%%%%%%%%%%%%%%%%%%%%%%%%%%%%%%%%%%%%%%%%%%%%%%%%%%%%%%%%%%%%%%%%%%%%%%

\chapter{Analytical estimate of the outgoing current for a FP process}
\label{app_sec:analytic_estimate_of_the_current_loss}

%%%%%%%%%%%%%%%%%%%%%%%%%%%%%%%%%%%%%%%%%%%%%%%%%%%%%%%%%%%%%%%%%%%%%%%%%%%%%%%%

We are interested in finding an accurate analytical approximation for the outgoing current of a FP process like Eq.~\eqref{eq:fp}. We start  by applying the following change of variables
\begin{equation}
    x = -\int_I^{I_\mathrm{a}} \frac{1}{D^{1/2}(I')}\,\mathrm{d}I'\,,\quad \rho_x(x,t)=\rho(I,t)\frac{\mathrm{d}I}{\mathrm{d}x}=\rho(I,t)\sqrt{D(I)} \, ,
\end{equation}
which leads to
\begin{equation}
    \pdv{\rho_x}{t} = \frac{1}{2}\pdv{x} \left[\frac{1}{D^{1/2}}\dv{D^{1/2}}{x}\rho_x\right]+\frac{1}{2}\pdv[2]{\rho_x}{x}\, ,
\end{equation}
where $D=D\left(I(x)\right)$. By introducing the effective potential $V(x)=-\ln(D^{1/2}(x))$, we obtain the Smoluchowsky form~\cite{hannes1996fokker}
\begin{equation}
    \pdv{\rho_x}{t} = \frac{1}{2}\pdv{x} \dv{V(x)}{x} \rho_x+\frac{1}{2}\pdv[2]{\rho_x}{x}\, .
    \label{eq:smol}
\end{equation}
Equation~\eqref{eq:smol} can be made self-adjoint by means of the following change of variables 
\begin{equation}
    %\rho'(x,\tau) = \exp\left[-\frac{V(x)}{2D}\right]p(x,\tau) \, ,
    \rho_x(x,t) = \exp\left[-\frac{V(x)}{2}\right]p(x,t) \, ,
\end{equation}
and Eq.~\eqref{eq:smol} is cast into the following form
\begin{equation}
    \pdv{p}{t} = \frac{1}{4}\left[\dv[2]{V}{x} - \frac{1}{2}\left(\dv{V}{x}\right)^2\right]p + \frac{1}{2}\pdv[2]{p}{x}\,.
    \label{eq:self-adj}
\end{equation}

The general solution of Eq.~\eqref{eq:self-adj} can be written as
\begin{equation}
    p(x,t) = \sum_\lambda c_\lambda(t)\phi_\lambda(x)\,,
    \label{eq:expansion}
\end{equation}
where an expansion using the eigenfunctions $\phi_\lambda(x)$ of the operator on the r.h.s.\ of Eq.~\eqref{eq:self-adj} has been used, namely 
\begin{equation}
    2\left\{-\frac{1}{4}\left[\dv[2]{V}{x} - \frac{1}{2}\left(\dv{V}{x}\right)^2\right] - \lambda \right\}\phi_\lambda(x) = \dv[2]{\phi_\lambda}{x}\,,
    \label{eq:eigenproblem}
\end{equation}
and $c_\lambda(t) = c_\lambda(0)e^{-\lambda t}$. This choice of eigenfunctions is motivated by the working hypothesis that $p(x,t\to+\infty)=0$, i.e.\ the system will eventually relax to a zero distribution.

By using the orthogonality and completeness properties of $\phi_\lambda(x)$,
\begin{align}
    \int \phi_\lambda(x)\phi_\lambda(x')\,\mathrm{d}x &= \delta(\lambda - \lambda')\\
    \sum_\lambda \phi_\lambda(x)\phi_\lambda(x') &= \delta(x-x')\,,
\end{align}
and considering the initial condition
\begin{align}
    \rho_x(x,0) &= \exp\left[-\frac{V(x)}{2}\right] p(x,0) \\
    p(x,0) &= \sum_\lambda c_\lambda(0) \phi_\lambda(x)\,, \label{eq:mid_step_analytic}
\end{align}
we have that
\begin{align}
    c_\lambda(0) = \int \exp\left[\frac{V(x)}{2}\right] \rho_x(x,0)\phi_\lambda(x)\, \mathrm{d}x\,,
    \label{eq:c_lambda_zero}
\end{align}
and the solution for an initial Dirac delta distribution $\rho_x(x, 0)=\delta(x-x_0)$ can be written as
\begin{equation}
    \rho_x(x,t)=\exp\left[\frac{V(x_0)-V(x)}{2}\right]\sum_\lambda e^{-\lambda t}\phi_\lambda(x_0)\phi_\lambda(x)\, ,
\end{equation}
and the outgoing current at an absorbing boundary in $x=0$, which in the original variables corresponds to $I=I_\mathrm{a}$, reads
\begin{equation}
    J(t) = \frac{1}{2}\pdv{\rho_x}{x}\Bigr|_{(0,t)}     \, .
    \label{eq:eigencurrent}
\end{equation}

If the potential is linearised, i.e.\ $V(x)\simeq -\nu \, x$, then there is an analytic solution to the eigenvalue problem in Eq.~\eqref{eq:eigenproblem}
\begin{equation}
    -2\left[\lambda-\frac{\nu^2}{2}\right]\phi_\lambda(x)=\dv[2]{\phi_\lambda}{x}\, ,
\end{equation}
and if we replace this solution in Eq.~\eqref{eq:eigencurrent}, we obtain the expression for the outgoing current
\begin{equation}
    J(x_0, t) = \frac{|x_0|}{t\sqrt{2\pi t}}\exp\left(-\frac{(x_0+\frac{\nu}{2}t)^2}{2t}\right) \,,
    \label{eq:out_current}
\end{equation}
which has dimension $t^{-1}$. Furthermore, the linearisation $\nu$ of the potential $V(x)$ near $x=x_0$ reads
\begin{equation}
    \nu=\frac{\frac{1}{2\kappa}}{I(x_0)}\left(\frac{I_\ast}{I(x_0)}\right)^{\frac{1}{2\kappa}}\exp\left[-\left(\frac{I_\ast}{I(x_0)}\right)^{\frac{1}{2\kappa}}\right]\,,
\end{equation}
which can be inserted into Eq.~\eqref{eq:thecurrent}, for obtaining an analytical estimate of the outgoing current.

%%%%%%%%%%%%%%%%%%%%%%%%%%%%%%%%%%%%%%%%%%%%%%%%%%%%%%%%%%%%%%%%%%%%%%%%%%%%%%%%

\chapter{Outgoing current for a system with infinite source}\label{app_sec:outgoing_current_for_a_system_with_infinite_source}

%%%%%%%%%%%%%%%%%%%%%%%%%%%%%%%%%%%%%%%%%%%%%%%%%%%%%%%%%%%%%%%%%%%%%%%%%%%%%%%%

To make use of the analytical  estimate of the outgoing current presented in Appendix~\ref{app_sec:analytic_estimate_of_the_current_loss}, we need to slightly modify certain steps to adapt to the different non-zero equilibrium distribution $\rho_\text{eq}$, as the original calculations are carried out under the assumption that $\rho(I,t\to+\infty)=0$, and modifications to Eq.~\eqref{eq:mid_step_analytic} need to be made and then propagated.

Under these new conditions, the expansion of the solution of the diffusive problem in Eq.~\eqref{eq:expansion} can be modified according to 
\begin{equation}
    p(x, t) = \sum_\lambda c_\lambda(t) \phi_\lambda(x) + \exp\left[\frac{V(x)}{2}\right] \rho'_\text{eq}(x)\,,
    \label{eq:new_expansion}
\end{equation}
where $\rho'_\text{eq}(x) = \rho_\text{eq}(I(x))\dv{I}{x}$ is the equilibrium distribution of our system, while considering the change of variables necessary to work with the self-adjoint diffusive problem in the Smoluchowsky form. The various considerations about $c_\lambda(t)$ and $\phi_\lambda(x)$ are unchanged. The values $c_\lambda(0)$ should be recomputed and from the expansion in Eq.~\eqref{eq:new_expansion}, we obtain
\begin{align}
    \rho'(x, 0) &= \exp\left[-\frac{V(x)}{2}\right]p(x,0)\\
    p(x, 0) &= \sum_\lambda c_\lambda(t) \phi_\lambda(x) + \exp\left[\frac{V(x)}{2}\right] \rho'_\text{eq}(x)\,,
\end{align}
which then leads to
\begin{equation}
    c_\lambda(0) = \int \exp\left[\frac{V(x)}{2}\right] \left\{\rho'(x,0) - \rho'_\text{eq}(x)\right\}\phi_\lambda(x)\, \mathrm{d}x = \int \exp\left[\frac{V(x)}{2}\right] \rho^\ast(x,0)\phi_\lambda(x)\, \mathrm{d}x\,,
\end{equation}
where, $\rho^\ast(x,t)$ stands for the difference between the actual and the equilibrium distribution, still to be reached, and in this framework, the rest of the analytic current estimate, i.e.\ Eq.~\eqref{eq:out_current}, still applies.
