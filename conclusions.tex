\chapter*{Conclusions}
\addcontentsline{toc}{chapter}{Conclusions}  
\chaptermark{Conclusion}

And that's a wrap folks! We've come to the end of this thesis and I hope you've enjoyed reading it as much as I enjoyed writing it.

Before you close the thesis and move on to your next review, I want to ask for a small favor. If you found this thesis helpful or informative, \textbf{please consider subscribing to my ORCID.} And if you really loved it, \textbf{why not smash that cite button?} Your support means the world to me and helps me reach more people and spread diffusion.

Thanks for reading and don't forget to hit that subscribe and cite button!

In this thesis, we reviewed a novel diffusive framework for describing the long-term behaviour of the betatron motion in circular accelerators. The framework consists of a Fokker-Planck model with a diffusive coefficient having a functional form related to the stability-time estimate of the Nekhorohsev theorem. Such a framework has the purpose to provide better insights on the evolution of the beam distribution, as well as to provide a better understanding on the formation of beam halo and of the scaling laws governing beam losses.

To test the functional form of the diffusion coefficient, consistent with the stability time estimate of the Nekhoroshev theorem, we have proposed and scrutinised through detailed numerical simulations an optimal measurement protocol, which uses collimator scans with a specific pattern and timing. This protocol consists of separating the measured loss signal into a global current and a recovery current, and reconstructing the global current to normalise the recovery current. The performance of the protocol was simulated in various configurations and found to be capable of reconstructing the parameters of the diffusion coefficient with good accuracy, especially when performed in a phase-space region where the diffusion coefficient has an exponential evolution. The protocol was also shown to provide useful information on possible shortcomings in the data set used for the analysis, such as a high uncertainty band in the global current reconstruction or a reconstructed value of $I_\ast$ that indicates that the probed phase space region is outside the optimal interval.

A variant of this protocol was then applied on existing LHC Run~2 collimator scan data, which were collected using a non-optimised protocol. Despite the non-ideal measurement protocol used, we were able to analyse BLM loss signals during collimator scans and obtain a promising reconstruction of recovery currents, indicating a behaviour that is compatible with a Nekhoroshev-like diffusive dynamics. To address missing information in the dataset, we adapted key features of our fitting procedure, leading to good reconstruction performance and insight into the global diffusive behaviour of the LHC beam halo. As some collimator scans carried out in the LHC 2022 run were performed using our proposed optimal protocol, we expect to be able to more accurately characterise non-linear diffusive behaviour, which will be the outcome of future work.

Another application of this general diffusive framework was the analysis of the long-term effects of beam-beam wire compensators on beam losses. The promising results obtained in the various fit reconstructions provided, in general, positive insight on the beam-halo dynamics, suggesting that the wires do not have unwanted side effects on the beam dynamics and that our diffusive model might be a promising tool to assess these long-term effects. Here, as well, future studies will be performed using the recent data collected during the 2022 LHC run, which were collected using a better approach that allows keeping under control the beam-halo evolution.

Regarding the domain of single-particle effects, we considered the analysis of the performance of dynamic indicators aimed at detecting chaotic behaviour in the orbits of symplectic dynamical systems. After measuring the classification performance of various indicators on a modulated Hénon map, we have determined which are the best performing ones, in terms of correctly assessing the chaotic character of an orbit using a minimal number of turns. On the basis of the findings of these studies, the best indicators have been applied to the study of the beam dynamics in realistic magnetic lattices of the LHC luminosity upgrade.

This analysis provided some interesting elements on the connection between Nekhoroshev scaling laws for stability times and the presence of large chaotic regions in the phase space, in terms of Lyapunov time, measured by means of the $FLI^{WB}$ dynamic indicator. In fact, it has been possible to show that both the stability time and the Lyapunov time follow the same Nekhoroshev-like scaling law, although with different model parameters. Moreover, the $GALI$ dynamic indicator was used to first inspect its time dependence for chaotic orbits, while the analysis of the geometry of the orbits is a topic of future studies.

In general, the less known dynamic indicators explored in these studies provided interesting results that are worthy of further inspection and will eventually be included in the toolbox of an accelerator physicist.

The research line presented in this thesis still requires more steps and developments to be considered fully complete. As we look forward to the present and future measurement opportunities offered at the CERN LHC, we hope to be able to confirm and consolidate the initial promising results achieved here.

Collimator scans performed following our requirement, as well as new beam-beam wire compensator measurements, will provide new data to be studied and used to assess the consistency of our proposed diffusive framework. This, along with further developments, shall provide better insights into the complex topic of non-linear beam dynamics and beam-halo formation.

% \vspace{3em}

% \begin{flushright}
% \textit{Meyrin, 31 January 2023}
% \end{flushright}
