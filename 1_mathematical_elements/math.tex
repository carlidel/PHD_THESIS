
\chapter{Mathematical elements}\label{ch:mathematical_elements}

\section{Hamiltonian mechanics}\label{sec:1:hamiltonian}

\section{Averaging principle}\label{sec:1:averaging}
% qui a quale teorema ti riferisci? Dal momento che la tua tesi utilizza un approccio diffusivo alla dinamica Hamiltoniana io vedrei bene una parte che sul caos e indicatori correlati in modo da inserire l'approccio stocastico, In questo senso i teoremi delle media sarebbero sulle variabili stocastiche angolari non nel caso deterministico

\section{Kolmogorov-Arnold-Moser (KAM) theorem}\label{sec:1:kam}
% Questa parte dovrebbe essere breve e precisa (ovvero i teoremi vanno citati in modo rigoroso). La focalizzerei sulle mappe per le applicazioni agli acceleratori e sulla nascita delle aree debolmente caotiche. La facciamo a quattro mani (ovvero imposta il file tex e poi intervengo io)

\section{Nekhoroshev theorem}\label{sec:1:nekhoroshev}
%  Qui come sopra: il th di Nekhoroshev ha molte formulazioni (soprattutto per la questione dell'esponente nella stima ottimale). Io non entrerei nei dettagli ma nella generalità del risultato correlato all'analiticità delle serie perturbative


