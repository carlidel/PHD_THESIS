\chapter*{Report on PhD activities}
\addcontentsline{toc}{chapter}{Report on PhD activities}  
\chaptermark{Report on PhD activities}
\sectionmark{}

This PhD was performed under the joint supervision of Prof.~Armando Bazzani and Dr.~Massimo Giovannozzi, affiliated to the BE-ABP-NDC section at CERN. The first year of the PhD has been located at UNIBO, while the remaining years were spent in exchange at CERN within the BE-ABP-NDC section, under a Doctoral Student contract.

\section*{Scientific activity}

The line of research of this PhD has been mainly focused on two projects:
\begin{itemize}
    \item Formulation of an optimised measurement protocol for the LHC for
probing the non-linear diffusive behaviour of beam-tails via collimator
scans.
    \item Analysis of Dynamic Indicators for probing the chaotic behaviour of particles
in single-particle tracking simulations.
\end{itemize}

For the first line of work, the first two years were spent for the creation of the protocol, along with the formulation of simulation results and the writing of a first original paper for a peer-reviewed journal, documenting the proposed protocol and simulation results. Next, this
model was applied to available LHC collimator scan data.

An MD proposal has been proposed for executing such measurement
protocol on the first MD block of Run3. Due to unfortunate events that led to
the delay of various MD, we had to rely on occasional End of Fill
measurements (EoF). Some EoF measurement opportunities occurred between the months of October and December 2022, and the resulting data will be analysed in these first months of 2023.

For the second line of work, some first results were obtained in inspecting
the performance and implementation of Dynamic Indicators such as the
Reversibility Error and the Generalised Alignment Index in the context of GPU
parallelised single-particle tracking, and we are now looking forward to presenting
two separate papers on the archived results.

In addition to these two main lines of research, an initial collaboration was
started with the application of diffusive models to describe and inspect the
performance of beam-beam wire compensators. This collaboration might offer
insights into the long-term effects on beam-tail populations caused by wire
compensators.

\section*{Publications}

\begin{itemize}
    \item (In preparation) A.\ Bazzani, M.\ Giovannozzi, C.E.\ Montanari, G.\ Tur\-chet\-ti. \textit{``Chaos indicators and nonlinear dynamics in circular particle accelerators''}.
    
    \item (To be submitted) A.\ Bazzani, M.\ Giovannozzi, C.E.\ Montanari, G.\ Turchetti. \textit{``Performance analysis of indicators of chaos for nonlinear dynamical systems''}.

    \item  C.E.\ Montanari, A.\ Bazzani, M.\ Giovannozzi. \textit{``\href{https://doi.org/10.1140/epjp/s13360-022-03478-w}{Probing the diffusive behaviour of beam-halo dynamics in circular accelerators}''}. Eur. Phys. J. Plus (2022).

    \item M.\ Giovannozzi, E.H.\ Maclean, C.E. Montanari, G. Valentino, F.F.\ Van der Veken. \textit{``\href{https://doi.org/10.3390/info12020053}{Machine Learning Applied to the Analysis of Nonlinear Beam Dynamics Simulations for the CERN Large Hadron Collider and Its Luminosity Upgrade}''}. Information, 12(2), 53 -- 2020.

    % \item A.\ Bazzani, M.\ Giovannozzi, E.H.\ Maclean, C.E.\ Montanari, F.F.\ Van der Veken e W.\ Van Goethem. \textit{``\href{https://doi.org/10.1103/PhysRevAccelBeams.22.104003}{Advances on the modelling of the time evolution of dynamic aperture of hadron circular accelerators}''}. Phys. Rev. Accel. Beams 22, 104003 -- 24 October 2019.
    %\item A.\ Bazzani, S.\ Vitali, C.E.\ Montanari, M.\ Monti e G.\ Castellani. \textit{``\href{https://doi.org/}{Stochastic properties of colliding particles in a non-equilibrium thermal bath}''}. In: 2019. url:\url{ https://sites.google.com/view/lfo12-13aprile2019/programme}. (Abstract presented at the conference ``Nonlocal and Fractional Operators'', Rome, 12 April 2019).
\end{itemize}

\section*{Conference proceedings}

During this PhD, I had the opportunity to take part to the international conferences of IPAC'21 and IPAC'22. This has resulted in the following proceedings either as main author or coauthor:
 
\begin{itemize}
    \item C.E.\ Montanari, A.\ Bazzani, M.\ Giovannozzi, and G.\ Turchetti, \textit{``\href{
https://doi.org/10.18429/JACoW-IPAC2022-MOPOST042}{Using Dynamic Indicators for Probing Single-Particle Stability in Circular Accelerators}''}, in Proc. IPAC'22, Bangkok, Thailand, Jun. 2022, pp. 168-171.

    \item C.E.\ Montanari, A.\ Bazzani, M.\ Giovannozzi, A.A.\ Gorzawski, and S.\ Redaelli, \textit{``\href{https://doi.org/10.18429/JACoW-IPAC2022-MOPOST043}{Testing the Global Diffusive Behaviour of Beam-Halo Dynamics at the CERN LHC Using Collimator Scans}''}, in Proc. IPAC'22, Bangkok, Thailand, Jun. 2022, pp. 172-175.

    \item F.F.\ Van der Veken et al., \textit{``\href{https://doi.org/10.18429/JACoW-IPAC2022-MOPOST047}{Determination of the Phase-Space Stability Border with Machine Learning Techniques}''}, in Proc. IPAC'22, Bangkok, Thailand, Jun. 2022, pp. 183-186.

    \item C.E. Montanari, A. Bazzani, and M. Giovannozzi, \textit{``\href{https://doi.org/10.18429/JACoW-IPAC2021-TUPAB233}{Diffusive Models for Nonlinear Beam Dynamics}''}, in Proc. IPAC'21, Campinas, SP, Brazil, May 2021, pp. 1976-1979.

    \item F.F. Van der Veken, M. Giovannozzi, E.H. Maclean, C.E. Montanari, and G. Valentino, \textit{``\href{https://doi.org/10.18429/JACoW-IPAC2021-MOPAB028}{Using Machine Learning to Improve Dynamic Aperture Estimates}''}, in Proc. IPAC'21, Campinas, SP, Brazil, May 2021, pp. 134-137.
\end{itemize}

\section*{Internal talks and presentations}

\begin{itemize}
    \item \href{https://indico.cern.ch/event/1212187/#15-diffusion-mechanisms-includ}{Diffusion mechanisms including Xsuite development and collimation experiments} (ColUSM \#156: Special Joint HiLumi WP2/WP5 Meeting).
    \item \href{https://indico.cern.ch/event/1188011/#2-update-on-analysis-of-wire-c}{Update on analysis of wire compensators loss data} (Non-linear WG).     
    \item \href{https://indico.cern.ch/event/1162317/#21-update-analysis-of-lhc-diff}{Update analysis of LHC diffusion from LHC collimator scans} (BE-ABP-NDC section meeting).
    \item \href{https://indico.cern.ch/event/1100158/#2-probing-the-diffusive-behavi}{Probing the diffusive behaviour of beam-halo dynamics in circular accelerators} (Non-linear WG).
\end{itemize}

\section*{PhD courses}

\begin{itemize}
    \item Theoretical Physics:
    \begin{itemize}
        \item C.\ Degli Esposti Boschi, D.\ Vodola: Quantum Mechanics and Entanglement (3 CFU)
    \end{itemize}
    \item Nuclear Physics:
    \begin{itemize}
        \item R.\ Spighi, M.\ Franchini: Physics of Hadrontherapy (1.5 CFU)
        \item A. Gabrielli, Off-Detector Trigger and DAQ (1.5 CFU)
    \end{itemize}
    \item Materials Physics:
    \begin{itemize}
        \item T. Cramer: Probing material properties at the nano-scale with atomic force microscopy (1.5 CFU)
        \item L. Pasquini: Advanced Data Acquisition (1.5 CFU)
    \end{itemize}
    \item Applied Physics:
    \begin{itemize}
        \item G.\ Castellani, D.\ Remondini, M.\ Bersanelli: Statistical Learning and Selected Research Topics (3 CFU) 
    \end{itemize} 
\end{itemize} 

\section*{External training}

\begin{itemize}
    \item Joint Universities Accelerator School: The science of particle accelerators (January 2021 -- February 2021)
    \item INFN: Efficient Scientific Computing `21 (October 2021)
    \item CERN learning hub:  Scientific Writing (May 2022).
    \item CERN learning hub:  Convincing Scientific Presentations (September 2022).
    \item CERN learning hub:  Resilience for Researchers (September 2022 -- November 2022).
    
\end{itemize}