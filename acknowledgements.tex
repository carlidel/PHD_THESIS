\chapter*{Acknowledgements}
\addcontentsline{toc}{chapter}{Acknowledgements}  
\chaptermark{Acknowledgements}
\sectionmark{}

% \epigraph{
% \textit{????}
% }{\textsc{maybe.}}

The research and findings presented in this thesis would not have been possible without the support, guidance, and mentoring of many individuals. I am first and foremost grateful to my supervisors, Prof. Armando Bazzani at Bologna University and Dr. Massimo Giovannozzi at CERN. Their consistent support, willingness to assist, and unflagging patience have been invaluable to me since my introduction to the field of Accelerator Physics as a Summer Student in 2018. Their guidance has been a constant throughout the entire journey.

I then express my gratitude to the entire BE-ABP-NDC section at CERN for the warm welcome and constant support, offered along the various stages of this research work and, especially, during the very narrow data gathering opportunities which occurred in the last months of 2022 LHC operation. For this last part, a dedicated thanks is due to Dr.\ Stefano Redaelli and Dr.\ Pascal Dominik Hermes, for assisting and guiding me through the various crucial stages of data gathering. A sincere thank also goes to Dr. Frederik Van der Veken for both the multiple moments of mentoring and for the inclusion in engaging side projects. 

I also want to thank Dr.\ Axel Poyet and Dr.\ Guido Sterbini, for enabling the entire application of our diffusive framework on wire compensator losses, and giving us the opportunity of performing dedicated data gathering during the latest wire compensator MD session.

A dedicated thanks goes to Prof.\ Giorgio Turchetti, for the engaging teaching, the dedicated mentoring, and the research opportunity offered on the vast topic of dynamic indicators.

Prior to submitting the final version of this thesis, I would like to express my gratitude to Prof. Roberto Artuso and Dr. Stefano Redaelli. Their review of the preliminary edition of this work was greatly appreciated, and I am thankful for their thorough reading and motivating comments.

Contemporary research, and, arguably, research in general, cannot be carried forward by lonely individuals, but rather by a dynamic and motivated community of collaborating scientists. The scourging isolation brought on by the Covid pandemic has also highlighted the importance of maintaining these international communities active. I am grateful and honoured to have been included in such a community during this years, and to have had the opportunity to personally meet the wonderful and unique people that constitute it.

