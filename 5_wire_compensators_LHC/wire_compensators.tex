\chapter{Diffusive model application on wire compensators losses}

\section{LHC beam-beam wire compensators fundamentals}

One of the most significant limits in the present LHC design and future HL-LHC design is given by the electromagnetic interactions between the two counter-rotating beams in the shared sections of the machine, the interaction points~\cite{Arduini_2016}. These interactions lead to the so-called \textit{beam-beam effects}, and can be distinguished as head-on beam-beam effects, which happens when the beam bunches are overlapping in the interaction point, and long-range beam-beam effects, caused by having the two beams close together for the whole length of the shared region.

In the LHC, the beams are set in collision with a small \textit{crossing angle}, which has both the purpose to separate bunches immediately upstream and downstream of the collision point~\cite{Arduini_2016} and to reduce the presence of long-range beam-beam effects, as smaller angles imply smaller beam separation. However, an increase in crossing angle also implies a reduction in integrated luminosity, i.e.\ the total number of collisions that have occurred over a given period of time, typically measured in inverse femtobarns \SI{}{fb}$^{-1}$~\cite{Herr:941318}, as the bunches overlap decreases. A schematic visualization of two different crossing angles and their consequent effect on beam separation and bunch overlap is presented in Fig.~\ref{fig:crossing-angles}. The nominal full crossing angle for the LHC is set at $\theta_c =$ \SI{285}{\micro\radian}, while for HL-LHC the expected baseline parameter will be set at $\theta_c =$ \SI{500}{\micro\radian}~\cite{BejarAlonso:2749422}. 

\begin{figure}[hpt]
    \centering
    \def\svgwidth{1.0\textwidth}
    \import{5_wire_compensators_LHC/figs/}{crossing_angle_basics.pdf_tex}
    \caption{Schematic visualization of two different crossing angles and their consequent effect on beam separation and bunch overlap. As $\theta_{c1} > \theta_{c2}$, the bunches are more separated in the first case (left), while the bunch overlap is larger in the second case (right). The long-range beam-beam effects are also more significant in the second case.}
    \label{fig:crossing-angles}
\end{figure}

To tackle the long-range beam-beam effects, while maintaining a small crossing angle, a corrective approach, based on electro-magnet lenses, was presented in~\cite{Koutchouk:692058}. The core concept of this approach consists in compensating the perturbation pattern given by long-range effects by using DC wires. The idea of using specifically DC wires comes from the observation on how these long-range effects were similar to the $1/r$ dependence similar to the electric field potential given by a simple wire~\cite{PhysRevSTAB.5.074001}. A sketch of the concept of the compensation given by the BBCW is presented in Fig.~\ref{fig:wire-baseline}.

\begin{figure}[hpt]
    \centering
    \def\svgwidth{1.0\textwidth}
    \import{5_wire_compensators_LHC/figs/}{crossing_angle_kick.pdf_tex}
    \caption{Left, sketch of the long-range beam-beam effects felt by Beam~2 from Beam~1, following the weak-strong approximation (i.e.\ we consider the effect of Beam~1 on Beam~2 and not vice versa). Right, sketch of the principle of the compensation given by the BBCW. The long-range beam-beam effects (red arrow) are compensated by the DC wires (green arrow), which are placed in the beam path before and after the interaction point.}
    \label{fig:wire-baseline}
\end{figure}

The resulting tool is referred to as \textit{``beam-beam wire compensator''} (BBCW), and it has been tested in multiple iterations on various accelerator complex (a complete list of experimental applications, up to early 2022, is available in~\cite{axel.wires}).

In the LHC, the BBCW are installed for Beam~2 only, as it was the only beam foreseen to operate with a coronograph~\cite{Goldblatt:2313940}, which is a device that is expected to allow transverse beam halo measurements in the future. The wires are embedded in the tertiary collimators placed before and after IP1 and IP5~\cite{Rossi:2696270}. These collimators are still part of the collimator hierarchy, presented in Section~\ref{sec:collimation}, with the primary and secondary components placed in IR3 and IR7, and are placed near the IPs to locally protect the experiments from potentially damaging losses.

A collimator hosts two separate wires (one per jaw), and each wire can carry up to \SI{350}{\ampere}. These two separate wires are cabled in series such that they have the same polarity. This configuration enables a specific 2-jaws powering setup with the characteristics of doubling the odd multipolar strength of the kick, while the even ones cancel out. This choice is motivated by the need for a compensation of the octupolar resonances~\cite{Poyet:2703503}. The two possible configurations are presented in Fig.~\ref{fig:wire-configs}. The first one, referred to as \textit{single wire} configuration, only powers up the internal wires. The second one, referred to as \textit{quadrupolar configuration}, powers up both wires.

\begin{figure}[hpt]
    \centering
    \def\svgwidth{1.0\textwidth}
    \import{5_wire_compensators_LHC/figs/}{crossing_angle_configs.pdf_tex}
    \caption{Left, sketch of the single wire configuration, where only one wire per pair is powered up. Right, quadrupolar configuration, where both wires in the pairs are powered up to compensate the octupolar resonances.}
    \label{fig:wire-configs}
\end{figure}

\section{Overview of experimental data}

For the application of our diffusive framework, we consider the experimental data gathered at the CERN LHC during the 2018 LHC Machine Development (MD) program, during the measurements campaign of the BBCW~\cite{Poyet:2703503}. More specifically, we consider the data gathered during fill 7386, as it has the configuration closest to an operational scenario.

During this MD measurement, the BBCW prototypes installed for Beam~2 were tested in various long range beam-beam dominated scenarios. Starting from low intensity fills for validating the BBCW configuration, up to high-intensity fills, such as the one we are considering. 

During fill 7386, three trains of symmetric bunches were tested at flat top energy with beams colliding at different crossing angles, smaller than the nominal one in order to have strong long range beam-beam effects. During this operational-like fill, the wire compensator were set in the quadrupolar configuration. The loss signals measured by the BLMs for the two beams are displayed in Fig.~\ref{fig:wire-data}, note that the measurement unit is in \SI{}{protons \per s}, as we are considering calibrated losses. Along with the BLM data, the beam intensity measured by the BCTs, the wire power, the octupoles power, and the crossing angles are also reported. 

\begin{figure}[hpt]
    \centering
    % \includegraphics[]{}
    \caption{Overview of the data gathered during fill 7386. BLM calibrated losses and BCT beam intensity measurements were taken at different combinations of crossing angles, wire power, and octupoles power.}
    \label{fig:wire-data}
\end{figure}

The BLM data for Beam~1 and Beam~2 in units of \SI{}{protons \per s} represents a high-precision measure of protons lost over time. The BCT data, instead, provides a measurement of the intensity of the beam in number of protons over time. Such measurement is not particularly precise and, qualitatively, it can be seen from the plot how the intensity measured is does not distinguish the different regimes of losses highlighted by the BLM data, and maintains a steady linear decrease. This difference in precision makes the BLM data a fundamental tool for inspecting the different regimes of losses in the transverse plane.  

BBCW and octupole magnets state is reported in Ampere units. The BBCW wires can be found either in an \textit{off} state, namely at \SI{0}{\ampere}, or in an \textit{on} state, namely at \SI{350}{\ampere}. The octupoles instead are found at two different amperage values, namely $260$ and \SI{-560}{\ampere}. It is important to highlight how the switch of state for both systems is not instantaneous but requires instead a non-negligible amount of time, this will be thoroughly commented in the next section.

It is possible to see how the data provides a various number of crossing angles configurations, along with on-off alternations of both BBCW power and octupoles current. Qualitatively, one can see how the BBCW equipped in Beam~2 do lead to a lower BLM loss signal when they are turned on, while instead having them turned off leads to a strong peak in the losses measured by the BLMs.

We will now discuss how our diffusive framework can be applied to this loss signal, along with some necessary considerations on how one must pre-process and interpret this data, before applying the model. 

\section{Application of the diffusive model}

Let us consider the Fokker-Planck equation presented in Eq.~\eqref{}, with the Nekhoroshev-like functional diffusion coefficient of Eq.~\eqref{}. We recall that the system is fully characterized by the three free parameters $\epsilon$, $I_\ast$, and $\kappa$.

To apply this model to the BLM data, and reconstruct the $D(I)$ for the various states of the system, we perform a fitting approach inspired by the procedure used in the work of Bazzani et al.~\cite{bazzani2020diffusion}, where the same Fokker-Planck model is used to reconstruct the normalized intensity of the beam, evaluated over the number of turns.

Starting from the BLM and BCT data we have at hand, we want to construct a measure of the normalized intensity of the beam over the number of turns. To archive this, we first convert the measurements from seconds to number of turns, considering that, in LHC Run2, a reference proton at \SI{6.5}{TeV} performs 11245 turns every second. To then evaluate the relative intensity lost over an interval $[N_0, N_1]$, as the BCT data is not precise enough to highlight fine differences in loss rates, we consider as amount of protons lost the integrated BLM signal, and we take the BCT value registered at $N_0$ as the reference intensity. This procedure is illustrated in Fig.~\ref{fig:blm-to-intensity}.

\begin{figure}[hpt]
    \centering
    % \includegraphics[]{}
    \caption{Example of the procedure used to evaluate the relative intensity lost over an interval $[N_0, N_1]$. The BLM signal is integrated over the interval, and the BCT value registered at $N_0$ is used as the reference intensity.}
    \label{fig:blm-to-intensity}
\end{figure}

Along the various assumptions made in order to enable the application of the Nekhoroshev estimate, along with the averaging principle on the angular variables, it is important to highlight that this model makes the strong hypothesis that $D(I)$ does not evolve over time. This implies that the magnetic lattice of the accelerator must not manifest stronger variations than the ones given by the small stochastic perturbation. Such strong assumption requires some preliminary consideration on the BLM loss signal we have at hand.

When the BBCW, the crossing angle, and the octupoles are in a stationary state, we can state that the parameters of the Fokker-Planck equation can be considered as constant over time, since the magnetic lattice is not undergoing significant variations. We can define this state as stationary, and have the beam distribution $\rho(I, t)$ following the evolution defined by the Fokker-Planck.

When instead a variation in any of the elements happen, e.g.\ a BBCW is switched on or the crossing angle is varied, we have the parameters of the Fokker-Planck equation will vary as well into a new value. Such variation does not happen necessarily in a negligible time: as it can be seen in Fig.~\ref{fig:transient-state}, the BBCW DC current being switched on to the target voltage takes a significant number of seconds, leading to a time interval in which the system is undergoing a transient state. During such transient state, we can not make assumptions on the value evolution of $\epsilon$, $I_\ast$, and $\kappa$, and we are therefore forced to discard these slices of data.

\begin{figure}[hpt]
    \centering
    \caption{Visualization of the difference between stationary state and transient state for a slice of data. We define as stationary state the time interval in which all the parameters of the system are in a steady state, while the transient state is the time interval in which the system is undergoing a variation. Here, the BBCW DC current is switched on to the target voltage, causing a transient state along the process.}
    \label{fig:transient-state}
\end{figure}

In Fig.~\ref{fig:chunks}, we show the BLM data for Beam~1 and Beam~2 divided into enumerated chunks where the system is in stationary state. We can see how the stationary states are in general longer than the transient states, except for the part where octupoles are changed in state.

\begin{figure}[hpt]
    \centering
    % \includegraphics[]{}
    \caption{Experimental data divided in chunks, where each chunk is in a stationary state. Each chunk is characterized by a different set of parameters, and the system is in a transient state when the parameters are changed. Each chunk has a number assigned, which will be used to identify it for the rest of the analysis.}
    \label{fig:chunks}
\end{figure}

We assume that the beam distribution at the end of a stationary state can be used as the initial condition of the next stationary state. Therefore, we completely neglect the transient state losses between the two stationary states. To justify this approach, however, we must have that the integrated loss in transient state must not be significantly comparable to the integrated losses in stationary state.

In Fig.~\ref{fig:wire_loss_comp}, such comparison is displayed for each individual chunk, and it is possible to see how the interval where the octupoles state is changed has higher relative transient losses. Such comparable losses led us to the decision to not inspect this chunk of data characterized by varying octupoles, and we performed our fitting procedure only to the data up to those variations.

\begin{figure}[hpt]
    \centering
    % \includegraphics[]{}
    \caption{Comparison of the integrated losses in transient state and in stationary state for each chunk of data. The chunk numbers follow the nomenclature defined in Fig.~\ref{fig:chunks}. A higher relative loss in the transient states can be seen for the chunk where the octupoles are changed.}
    \label{fig:wire_loss_comp}
\end{figure}

Now that we have defined the data we will use for the fitting procedure, we can define the initial beam distribution we will use for the fitting. We consider a Gaussian beam distribution with unitary $\sigma$, which in action variables reads as a negative exponential distribution $\rho_0(I) = \exp(-I)$. To fit the data, we use the same procedure as in the work of Bazzani et al.~\cite{bazzani2020diffusion}, i.e.\ we perform a scan over the $\kappa$ and $I_\ast$ values, and we integrate the evolution of the FP equation over the number of turns. The $\epsilon^2$ parameter is then fixed by requiring that the initial and final value of the relative intensity, evaluated at the beginning and at the end of the chunk, are equal.

As we are assuming that the beam distribution at the end of a stationary state can be used as the initial condition of the next stationary state, we can use the evolved beam distribution as the initial condition for the next chunk. This procedure, iterated for all chunks, finally gives us the reconstructed $D(I)$ for the various states of the system.

\section{Numerical results}

We performed the fitting procedure for the data of both Beam~1 and Beam~2. As the wires are installed on Beam~2 only, we do not expect to observe a significant difference between the $D(I)$ reconstructed on the Beam~1 chunks with wire on and wire off. However, the data of Beam~1 is useful for both understanding the effects of different crossing angles on the system and to establish the characteristics of the results provided by the fitting procedure.

\subsection*{Beam~1 data}

We fist consider the data of Beam~1 divided in chunks only where the crossing angle is varied. In Fig.~\ref{fig:reconstruction_1}, we show the relative intensity loss, along with the fit reconstruction. Note that we considered Beam~1 data up to the point in which the octupoles current is varied. In Fig.~\ref{fig:reconstruction_2}, we show the reconstructed $D(I)$ for the various crossing angles.

We can see how the $D(I)$ increases as the crossing angle is lowered, as expected from the fact that the long-range beam-beam effects are more important, and we can also observe how in general the fitting procedure manages to reproduce the data quite well. The evolution of the three parameters, $\epsilon$, $I_\ast$, and $\kappa$, is shown in Fig.~\ref{fig:parameters_1}. No significant patterns can be observed in the evolution of the parameters, however, the resulting $D(I)$ does increase as the crossing angle is lowered.

\begin{figure}[hpt]
    \centering
    % \includegraphics[]{}
    \caption{Relative loss of intensity for the data of Beam~1 divided in chunks where the crossing angle is varied. The fit reconstruction is also shown. A good agreement between the data and the fit is observed.}
    \label{fig:reconstruction_1}
\end{figure}

\begin{figure}[hpt]
    \centering
    % \includegraphics[]{}
    \caption{Reconstructed $D(I)$ for the data of Beam~1 divided in chunks where the crossing angle is varied. The $D(I)$ increases as the crossing angle is lowered.}
    \label{fig:reconstruction_2}
\end{figure}

\begin{figure}[hpt]
    \centering
    % \includegraphics[]{}
    \caption{Evolution of the three parameters, $\epsilon$, $I_\ast$, and $\kappa$, for the data of Beam~1 divided in chunks where the crossing angle is varied. No significant patterns can be observed in the evolution of the parameters.}
    \label{fig:parameters_1}
\end{figure}

If we instead consider the data of Beam~1 divided in chunks where the wire compensators are switched on and off, we can see how the $D(I)$ reconstructed, presented in Fig.~\ref{fig:reconstruction_3}, while manifesting differences, does not show a consistent different diffusion value between the two states. This suggests that, indeed, the BBCW on Beam~2 is not significantly affecting the beam dynamics on Beam~1.

The evolution of the three parameters, $\epsilon$, $I_\ast$, and $\kappa$, is shown in Fig.~\ref{fig:parameters_2}. No significant patterns can be observed in the evolution of the parameters.

A comparison of the $\chi^2$ values of the fit for the two different chunking methods is shown in Fig.~\ref{fig:chi2}. We can see how the $\chi^2$ values are mostly lower when smaller chunks are considered, but not consistently so. This suggests that, while no consistent differences can be observed in the reconstructed $D(I)$ between wire on and wire off states, the fitting procedure might be affected by overfitting when the chunks are too small.

\begin{figure}[hpt]
    \centering
    % \includegraphics[]{}
    \caption{Reconstructed $D(I)$ for the data of Beam~1 divided in chunks where the wire compensators are switched on and off. The $D(I)$ does not show a consistent different diffusion value between the two states. The numbers follow the nomenclature in Fig.~\ref{fig:chunks}.}
    \label{fig:reconstruction_3}
\end{figure}

\begin{figure}[hpt]
    \centering
    % \includegraphics[]{}
    \caption{Evolution of the three parameters, $\epsilon$, $I_\ast$, and $\kappa$, for the data of Beam~1 divided in chunks where the wire compensators are switched on and off. No significant patterns can be observed in the evolution of the parameters.}
    \label{fig:parameters_2}
\end{figure}

\begin{figure}[hpt]
    \centering
    % \includegraphics[]{}
    \caption{Comparison of the $\chi^2$ values of the fit for the two different chunking methods. The $\chi^2$ values are mostly lower when smaller chunks are considered, but not consistently so. The numbers follow the nomenclature in Fig.~\ref{fig:chunks}.}
    \label{fig:chi2}
\end{figure}

\subsection*{Beam~2 data}

We now inspect the data of Beam~2, where the wires are installed. We consider the data divided in chunks as presented in Fig.~\ref{fig:chunks}. In Fig.~\ref{fig:reconstruction_4}, we show the relative intensity loss, along with the fit reconstruction. In Fig.~\ref{fig:reconstruction_5}, we show the reconstructed $D(I)$ for the various crossing angles and the various wire states.

It can be seen how, in general, the fit reconstruction is able to reproduce the data quite well. Moreover, it is possible to see how the reconstructed $D(I)$ is consistently different when the wires are switched on and off, with in general higher diffusion values when the wires are off. This is in agreement with the expectation that the wires are able to reduce the long-range beam-beam effects, and thus the diffusion. Moreover, it is possible to see how such reconstructed $D(I)$ for wire on states has also lower values for low $I$ amplitudes. This suggests that indeed the BBCWs might provide a better long-term stability of the beam, without causing emittance growth.

The values of the three parameters, $\epsilon$, $I_\ast$, and $\kappa$, are shown in Fig.~\ref{fig:parameters_3}. Also in this case, no significant patterns can be observed in the evolution of the parameters for the different states of the system.

To quantify the effects of the wires on the long-term beam dynamics, we can take the various reconstructed $D(I)$ for one of the crossing angles, and consider the evolution of the relative intensity and emittance of a Gaussian distribution with a Fokker-Planck evolution over a number of turns of a higher order of magnitude. In Fig.~\ref{fig:evolution}, we show the evolution of the relative intensity and emittance for the various $D(I)$ reconstructed for the crossing angle of \SI{150}{\micro\radian}. We can see how the wires are able to archive both a lower relative intensity and a lower emittance.

\begin{figure}[hpt]
    \centering
    % \includegraphics[]{}
    \caption{Relative intensity loss and fit reconstruction for the data of Beam~2 divided in chunks, following the nomenclature of Fig.~\ref{fig:chunks}. 
    The fit reconstruction is able to reproduce the data quite well.}
    \label{fig:reconstruction_4}
\end{figure}

\begin{figure}[hpt]
    \centering
    % \includegraphics[]{}
    \caption{Reconstructed $D(I)$ for the data of Beam~2 divided in chunks, following the nomenclature of Fig.~\ref{fig:chunks}. It can be seen how the reconstructed $D(I)$ is consistently different when the wires are switched on and off, with in general higher diffusion values when the wires are off.}
    \label{fig:reconstruction_5}
\end{figure}

\begin{figure}[hpt]
    \centering
    % \includegraphics[]{}
    \caption{Evolution of the three parameters, $\epsilon$, $I_\ast$, and $\kappa$, for the data of Beam~2 divided in chunks, following the nomenclature of Fig.~\ref{fig:chunks}. No significant patterns can be observed in the evolution of the parameters.}
    \label{fig:parameters_3}
\end{figure}

\begin{figure}[hpt]
    \centering
    % \includegraphics[]{}
    \caption{Evolution of the relative intensity and emittance for the various $D(I)$ reconstructed for the crossing angle of \SI{150}{\micro\radian}. The evolution is obtained by simulating a Fokker-Planck process, with a Gaussian distribution as initial condition. According to the numerical integration, the wires are able to archive both a lower relative intensity and a lower emittance.}
    \label{fig:evolution}
\end{figure}

\section{Final remarks}

We have performed an initial study of the effects of the BBCWs on the long-term beam dynamics of the LHC using our diffusive framework. We have used the data of the LHC Beam~1 and Beam~2, taken during an MD measurement campaign of Run2, and we have reconstructed the $D(I)$ of various system configurations. Ultimately, we have found that the wires are able to reduce the long-range beam-beam effects, and thus the diffusion. Moreover, we have found that the wires are able to archive consistently a lower intensity loss and a lower beam emittance. As expected, Beam~1 is not affected significantly by the BBCWs on Beam~2.

As pointed out in the overview of the experimental data, a large portion of the data had to be discarded due to the large amount of losses happening during transient periods. Moreover, multiple chunks considered in the analysis are characterized by a very short time span, which might be too short to be representative of the long-term beam dynamics. This is a limitation of the data, which might have had a significant impact on the results.

However, the promising results obtained in the fit reconstruction seem to suggest that this diffusive framework can provide some insight into the long-term effects of BBCWs on the beam dynamics. This has motivated us to consider a different data gathering strategy for the scheduled MD measurements of LHC Run3. More specifically, we planned to measure the BLM losses while keeping the BBCWs on and off for longer time spans, in order to be able to better characterize the long-term effects of the wires.

In addition, we planned to perform collimation scans with the BBCWs on and off, in order to directly inspect the beam tail population and measure $D(I)$ following the protocol presented in Chapter~\ref{ch:probing}. With the intent of comparing the values obtained with the two different methodologies.

Both of these strategies in the data gathering have been successfully implemented during the MD measurement campaign of LHC Run3~\cite{}, and the results will be presented in a future publication.