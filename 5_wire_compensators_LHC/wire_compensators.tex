\chapter{Diffusive model application on wire compensators losses}

\section{LHC beam-beam wire compensators fundamentals}

One of the most significant limits in the present LHC design and future HL-LHC design is given by the electromagnetic interactions between the two counter-rotating beams in the shared sections of the machine, the \textit{interaction points}. These interactions lead to the so-called \textit{beam-beam effects}, and can be distinguished as head-on beam-beam effects, which happens when the beam bunches are overlapping in the interaction point, and long-range beam-beam effects, caused by having the two beams close together for the whole length of the shared region.

To tackle the long-range beam-beam effects, a corrective approach, based on electro-magnet lenses, was presented in~\cite{}, and further developed in~\cite{}. The idea of this approach is to compensate the perturbation pattern given by long-range effects by using DC wires. The idea of using specifically DC wires comes from the observation on how these long-range effects were similar to the $1/r$ dependence similar to the electric field potential given by a simple wire~\cite{}.

The resulting equipment is referred to as \textit{``beam-beam wire compensator''} (BBCW), and it has been tested in multiple iterations on various accelerator complex (a complete list of experimental applications, up to early 2022, is available in~\cite{}).

\section{Overview of experimental data}

For the application of our diffusive framework, we consider the experimental data gathered at the CERN LHC during the 2018 LHC Machine Development (MD) program, during the measurements campaign of the BBCW~\cite{}. More specifically, we consider the data gathered during fill 7386, as it has the configuration closest to an operational scenario.

During this MD measurement, BBCW prototypes were installed for Beam2 on both sides of IP1 and IP5 which are, respectively, the location of the ATLAS~\cite{} and CMS~\cite{} experiments. These prototypes were embedded in the jaws of Beam2 tertiary collimators, for a total of 8 wires installed. The technical details of such implementation are reported in~\cite{}.

A fill with three trains of symmetric bunches was then loaded inside the LHC and tested at flat top energy with beams colliding at different crossing angles. During this operational-like fill, the quadrupolar setup for the wire compensator was set. The loss signals measured by the Beam Loss Monitors (BLMs)~\cite{} for the two beams are displayed in Fig.~\ref{fig:wire-data}. Along with the BLM data, the beam intensity measured by the Beam Current Transform systems (BCTs)~\cite{}, the wire power, the octupoles power, and the crossing angles are also reported. 

\begin{figure}
    \centering
    % \includegraphics[]{}
    \caption{}
    \label{fig:wire-data}
\end{figure}

The BLM data for Beam1 and Beam2 is reported in units of \SI{}{p/s}, and represents a high-precision measure of protons lost over time. Such measure is achieved by considering a calibration parameter on the \SI{}{gy/s} data directly measured by the BLM system.

The BCT data, instead, provides a measurement of the intensity of the beam in number of protons over time. Such measurement is not particularly precise and, qualitatively, it can be seen from the plot how the intensity measured is does not distinguish the different regimes of losses highlighted by the BLM data, and maintains a steady linear decrease.

BBCW and octupole magnets state is reported in Ampere units. The BBCW wires can be found either in an \textit{off} state, namely at \SI{0}{\ampere}, or in an \textit{on} state, namely at \SI{350}{\ampere}. The octupoles instead are found at two different amperage values, namely $260$ and \SI{-560}{\ampere}. It is important to highlight how the switch of state for both systems is not instantaneous but requires instead a non negligible amount of time, this will be thoroughly commented in the next section.

It is possible to see how the data provides a various number of crossing angles configurations, along with on-off alternations of both BBCW power and octupoles current. Qualitatively, one can see how the BBCW equipped in Beam2 do lead to a lower loss signal when they are turned on, while instead having them turned off leads to a strong peak in losses.

We will now discuss how our diffusive framework can be applied to this loss signal, along with some necessary considerations on how one must pre-process and interpret this data, before applying the model. 

\section{Application of the diffusive model}

Let us consider the Fokker-Planck equation presented in Eq.~\eqref{}, with the Nekhoroshev-inspired functional diffusion coefficient of Eq.~\eqref{}. We recall that the system is fully characterised by the three free parameters $\epsilon$, $I_\ast$, and $\kappa$.

To apply this model to the BLM data, and reconstruct the $D(I)$ for the various states of the system, we start by considering an initial Gaussian beam distribution with unitary $\sigma$, which in action variables reads as a negative exponential distribution $\rho_0(I) = \exp(-I)$. Next, we perform a fit on the three free parameters in order to best reconstruct the relative intensity loss, which is constructed starting from the BLM data, as the BCT data is not precise enough.

Along the various assumptions made in order to enable the application of the Nekhoroshev estimate, along with the averaging principle on the angular variables, it is important to highlight that this model makes the strong hypothesis that $D(I)$ does not evolve over time. This implies that the magnetic lattice of the accelerator must not manifest stronger variations than the ones given by the small stochastic perturbation. Such strong assumption requires some preliminary consideration on the BLM loss signal we have at hand.

When the BBCW, the crossing angle, and the octupoles are in a stationary state, we can state that the parameters of the Fokker-Planck equation can be considered as constant over time, since the magnetic lattice is not undergoing significant variations. We can define this state as stationary, and have the beam distribution $\rho(I, t)$ following the evolution defined by the Fokker-Planck.

When instead a variation in any of the elements happen, e.g.\ a BBCW is switched on or the crossing angle is varied, we have the parameters of the Fokker-Planck equation will vary as well into a new value. Such variation does not happen necessarily in a negligible time: as it can be seen in Fig.~\ref{fig:transient-state}, the BBCW DC current being switched on to the target voltage takes a significant number of seconds, leading to a time interval in which the system is undergoing a transient state. During such transient state, we can not make assumptions on the value evolution of $\epsilon$, $I_\ast$, and $\kappa$, and we are therefore forced to discard these slices of data.

\begin{figure}
    \centering
    % \includegraphics[]{}
    \caption{}
    \label{fig:transient-state}
\end{figure}

In Fig.~\ref{fig:chunks}, we show the BLM data for Beam1 and Beam2 divided into enumerated chunks where the system is in stationary state. We can see how the stationary states are in general longer than the transient states, except for the part where octupoles are changed in state.

\begin{figure}
    \centering
    % \includegraphics[]{}
    \caption{}
    \label{fig:chunks}
\end{figure}

We assume that the beam distribution at the end of a stationary state can be used as the initial condition of the next stationary state. Therefore, we completely neglect the transient state losses between the two stationary states. To justify this approach, however, we must have that the integrated loss in transient state must not be significantly comparable to the integrated losses in stationary state.

In Fig.~\ref{fig:wire_loss_comp}, such comparison is displayed for each individual chunk, and it is possible to see how the interval where the octupoles state is changed has higher relative transient losses. Such comparable losses led us to the decision to not inspect this chunk of data, and we performed our fitting procedure only to the data up to those changes.

\begin{figure}
    \centering
    % \includegraphics[]{}
    \caption{}
    \label{fig:wire_loss_comp}
\end{figure}



\section{Numerical results}

\section{Final remarks}