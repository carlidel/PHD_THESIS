\usepackage[LGR,T1]{fontenc}
\usepackage[utf8]{inputenc}

\newcommand{\textgreek}[1]{\begingroup\fontencoding{LGR}\selectfont#1\endgroup}

\newcommand\hmmax{0}
\newcommand\bmmax{0}

\usepackage{titlesec, blindtext, xcolor}
\definecolor{gray75}{gray}{0.75}
\definecolor{unibo}{RGB}{187, 46, 41}
\newcommand{\hsp}{\hspace{20pt}}
\newcommand{\hsps}{\hspace{10pt}}
\titleformat{\chapter}[hang]{\Huge\bfseries}{\thechapter\hsp\textcolor{unibo}{|}\hsp}{0pt}{\Huge\bfseries}
\titleformat{\section}[hang]{\Large\bfseries}{\thesection\hsps\textcolor{unibo}{|}\hsps}{0pt}{\Large\bfseries}
\titleformat{\subsection}[hang]{\large\bfseries}{\thesubsection\hsps\textcolor{unibo}{|}\hsps}{0pt}{\large\bfseries}

\usepackage[french,italian,english]{babel}
%\usepackage[british]{babel}
\usepackage{physics}
\usepackage{amssymb}
\usepackage{amsthm}
\usepackage{amsmath}
\usepackage{mathtools}
\usepackage{bm}% bold math
\usepackage{graphicx}
\usepackage{multirow}
%\usepackage[pdf]{pstricks}
%\usepackage{ps4pdf} \PSforPDF{ \usepackage{pstricks} }
%\usepackage{pst-pdf}
%\usepackage{epstopdf}
\usepackage{import}
\usepackage{xifthen}
\usepackage{pdfpages}
\usepackage{transparent}
\usepackage{booktabs}
\usepackage{makecell}
\usepackage{comment}
\usepackage{adjustbox}

\newcommand{\opm}[1]{{\color{red}#1}}
\newcommand{\av}[1]{\left<{#1}\right>}
\newcommand{\ie}{i.e.\ }
\newcommand{\Ie}{I.e.\ }
\newcommand{\eg}{e.g.\ }
\newcommand{\etc}{etc.\ }
\newcommand{\Eg}{E.g.\ } 
\newcommand{\viz}{\textit{viz.}\ } 
\newcommand{\resp}{resp.\ }
\newcommand{\wrt}{w.r.t.\ }
\newcommand{\cf}{cf.\ }
\newcommand{\lhs}{l.h.s.\ }
\newcommand{\rhs}{r.h.s.\ }
\newcommand{\ed}{\, .}
\newcommand{\ec}{\, ,}
\newcommand{\ham}{\mathcal{H}}
\newcommand{\map}{\mathcal{M}}
\newcommand{\eps}{\varepsilon}
\newcommand{\Eta}{\mathrm{H}}
\newcommand{\phii}[1]{\phi_{#1,\text{i}}}
\newcommand{\pr}{P}
\newcommand{\Ji}[1]{J_{#1,\text{i}}}
\newcommand{\Jf}[1]{J_{#1,\text{f}}}
\newcommand{\epsm}{\eps_\text{m}}
\newcommand{\epsmi}{\eps_\text{m,i}}
\newcommand{\epsmf}{\eps_\text{m,f}}
\newcommand{\epsh}{\eps_\text{h}}
\newcommand{\epshi}{\eps_\text{h,i}}
\newcommand{\epshf}{\eps_\text{h,f}}
\newcommand{\ScaleFig}{1.5}


%\usepackage[rmargin=]{geometry}
\newcommand{\dt}{\dot}
\newtheorem{definition}{Definition}
\newtheorem{theorem}{Theorem}
%\newcommand{\av}[1]{\left<#1\right>}
\newcommand{\bfield}{\mathbf{B}}
\newcommand{\ph}{\varphi}
\newcommand{\lap}{\nabla^2}
%\renewcommand{\grad}{\nabla}
%\renewcommand{\div}{\nabla\cdot}
\newcommand{\rot}{\curl}
\newcommand{\img}{\mathrm{i}}
\usepackage[pdftex,bookmarks=true,bookmarksopen=true,bookmarksnumbered=true,pdfdisplaydoctitle=true,hidelinks]{hyperref}
\usepackage[separate-uncertainty=true]{siunitx}
\DeclareSIUnit\clight{\text{\ensuremath{c}}}

\usepackage{epigraph}
\usepackage{bookmark}

\usepackage{csquotes}
\usepackage[style=numeric-comp, giveninits=true,maxcitenames=10,maxbibnames=4, sorting=none]{biblatex}

\renewcommand*{\mkbibnamegiven}[1]{\textsc{#1}}
\renewcommand*{\mkbibnamefamily}[1]{\textsc{#1}}

%giustificazione epigrafi
\renewcommand{\textflush}{flushright}
%\renewcommand{\epigraphflush}{flushright}

\usepackage{pgf,tikz,pgfplots}
%\pgfplotsset{compat=1.15}
%\usepackage{mathrsfs}
%\usetikzlibrary{arrows}
\usepackage{tikz-cd}
\usetikzlibrary{3d,arrows}
\usetikzlibrary{decorations.markings}
\usetikzlibrary{arrows.meta, bending, calc}
\pgfplotsset{compat=1.15}
%\usepackage{gnuplot-lua-tikz}
\usepackage[labelfont={sc, color=unibo},font={small}]{caption}
\usepackage{subcaption}
%\captionsetup[subfigure]{labelfont=rm}
\usepackage{indentfirst}

\usepackage{hvfloat}

%%%%FONT
\usepackage[p,osf,sups]{baskervillef}
%\usepackage[lf]{Baskervaldx}
\usepackage[varqu,varl,var0]{inconsolata}
\usepackage[scale=.95,type1]{cabin}
\usepackage[baskerville,vvarbb]{newtxmath}
\usepackage[cal=boondoxo]{mathalfa}

\usepackage{fancyhdr}
\pagestyle{fancy}
\renewcommand{\chaptermark}[1]{\markboth{#1}{}}
\renewcommand{\sectionmark}[1]{\markright{#1}}
\fancyhf{}
\fancyhead[RO,LE]{\bfseries\thepage}
\fancyhead[LO]{\nouppercase{\itshape\rightmark}}
\fancyhead[RE]{\nouppercase{\scshape\leftmark}}
\renewcommand{\headrulewidth}{0.5pt}
%\renewcommand{\footrulewidth}{0.5pt}
%\addtolenght{\headheight}{0.5pt}
\fancypagestyle{plain}{
	\fancyhead{}
	\renewcommand{\headrulewidth}{0pt}
}

\usepackage{xpatch}
\xpretocmd\headrule{\color{unibo}}{}{\PatchFailed}

\makeatletter
\renewcommand{\@epirule}{{\color{unibo} \rule[.5ex]{\epigraphwidth}{\epigraphrule}}}
\makeatother
% %\makeatletter

%\newcommand\frontmatter{%
%    \cleardoublepage
%  \@mainmatterfalse
%  \pagenumbering{roman}}

%\newcommand\mainmatter{%
%    \cleardoublepage
%  \@mainmattertrue
%  \pagenumbering{arabic}}

%\newcommand\backmatter{%
%  \if@openright
%      \cleardoublepage
%  \else
%    \clearpage
%  \fi
%  \@mainmatterfalse
%  }

%\makeatother

%\usepackage{newlfont}
\usepackage{color}
%\textwidth=450pt\oddsidemargin=0pt

% % %et al. in corsivo$
\renewbibmacro*{name:andothers}{% Based on name:andothers from biblatex.def
  \ifboolexpr{
    test {\ifnumequal{\value{listcount}}{\value{liststop}}}
    and
    test \ifmorenames
  }
    {\ifnumgreater{\value{liststop}}{1}
       {\finalandcomma}
       {}%
     \andothersdelim\bibstring[\emph]{andothers}}
    {}}
\renewbibmacro{in:}{}

\usepackage{rotating}
%\usepackage{subfig}

\usepackage{multirow}
\def\block(#1,#2)#3{\multicolumn{#2}{c}{\multirow{#1}{*}{$ #3 $}}}
\allowdisplaybreaks[1]

\usepackage{pgfornament}

\makeatletter
\newcommand{\ifenoughspace}[3]{%
  \@tempdimc\pagegoal \advance\@tempdimc-\pagetotal%
  \ifdim #1>\@tempdimc #3 \else #2\fi}


\newcommand{\parseparator}{\par{}\begin{center}\begin{tikzpicture}\node{\pgfornament[color=unibo,height=1em]{84}};\end{tikzpicture}\end{center}\par{} }

\newcommand{\appseparator}{\ifenoughspace{5em}{\par{}\vspace{1em}\begin{center}\begin{tikzpicture}\node{\pgfornament[color=unibo,height=3em]{70}};\end{tikzpicture}\end{center}\par{}\vspace{1em}}{}}


\newcommand{\chapterornament}[1]{\ifenoughspace{5em}{\par\vfill\par \begin{center}\begin{tikzpicture}\node{\pgfornament[color=unibo,height=3em]{#1} \pgfornament[color=unibo,height=3em,symmetry=v]{#1}};\end{tikzpicture}\end{center}\par\vfill\par}{}}



\newcommand{\finalornament}{\ifenoughspace{5em}{\par{}\vfill\par{}\begin{center}\begin{tikzpicture}\node{\pgfornament[color=unibo,height=5em]{68}};\end{tikzpicture}\end{center}\par{}\vfill\par{}}{}}

%\newcommand{\endchaporn}[1]{\par{}\begin{center}\begin{tikzpicture}\node{\pgfornament[color=black,height=3em]{#1}};
%\node{\pgfornament[color=black,height=3em]{#1}};
%\end{tikzpicture}\end{center}\par{}}


%%%subappendices
\newenvironment{chapterappendices}[1][Appendices]
{
\section*{#1}
\addcontentsline{toc}{section}{\textcolor{unibo}{\textit{#1}}}
\markright{#1}  
%\counterwithin{figure}{section}
%\counterwithin{table}{section}
\renewcommand\thesubsection{\thechapter.\textsc{\alph{subsection}}}
}
{
%\counterwithout{figure}{section}
%\counterwithout{table}{section}
\renewcommand\thesubsection{subsection}
} 


\usepackage{makecell}

\DeclareMathOperator{\Sf}{S}
\DeclareMathOperator{\Cf}{C}

\DeclareMathVersion{normal2}


\DeclareSourcemap{
  \maps[datatype=bibtex]{
    \map[overwrite]{
      \step[fieldsource=doi, final]
      \step[fieldset=url, null]
      \step[fieldset=eprint, null]
    }  
  }
}

\usepackage[activate={true,nocompatibility},final,expansion=false,protrusion=true,tracking=true,kerning=true,spacing=true,factor=1100]{microtype}
 \SetTracking{encoding={*}, shape=sc}{10}
 \microtypecontext{spacing=nonfrench}
 
 \setlength\headheight{15pt}
 
 \usepackage{lettrine}
 
 
 \newcommand{\mylettrine}[2]{\lettrine[lines=4,lhang=.1,findent=.1em,nindent=0em]{\color{unibo}#1}{\textls[100]{\color{unibo}#2}}} 
 
 \usepackage[percent]{overpic}


%%% TOC OPTIONS 
% \usepackage{tocloft}
% \renewcommand{\cftpartpresnum}{Part~}
% \renewcommand{\cftpartaftersnumb}{~\Large{\textcolor{gray75}{|}}\hsps}
% \renewcommand{\cftchapaftersnumb}{~{\textcolor{gray75}{|}}\hsps}

%\usepackage{titletoc}
%\renewcommand*\l@part[2]{%
%  \ifnum \c@tocdepth >-2\relax
%    \addpenalty{-\@highpenalty}%
%    \addvspace{2.25em \@plus\p@}%
%    \setlength\@tempdima{3em}%
%    \begingroup
%      \parindent \z@ \rightskip \@pnumwidth
%      \parfillskip -\@pnumwidth
%      {\leavevmode
%       \large \bfseries Part #1 \hfil
%       \hb@xt@\@pnumwidth{\hss #2% 
%                          \kern-\p@\kern\p@}}\par
%       \nobreak
%         \global\@nobreaktrue
%         \everypar{\global\@nobreakfalse\everypar{}}%
%    \endgroup
%  \fi}

\makeatletter
\def\@part[#1]#2{%
    \ifnum \c@secnumdepth >-2\relax
      \refstepcounter{part}%
      \addcontentsline{toc}{part}{\partname~\thepart\hspace{1em}#1}%
    \else
      \addcontentsline{toc}{part}{#1}%
    \fi
    \markboth{}{}%
    {\centering
     \interlinepenalty \@M
     \normalfont
     \ifnum \c@secnumdepth >-2\relax
       \huge\bfseries \partname\nobreakspace\thepart
       \par
       \vskip 20\p@
     \fi
     \Huge \bfseries #2\par}%
    \@endpart}
\makeatother

\renewbibmacro*{volume+number+eid}{%
  \printfield{volume}%
%  \setunit*{\adddot}% DELETED
%  \setunit*{\addnbthinspace}% NEW (optional); there's also \addnbthinspace
  \printfield{number}%
  \setunit{\addcomma\space}%
  \printfield{eid}}
\DeclareFieldFormat[article]{number}{\mkbibparens{#1}}


\AtBeginBibliography{%
    \DeclareFieldFormat{labelnumberwidth}{%
        {\color{unibo}\mkbibbrackets{#1}}%
    %    {\mkbibbrackets{#1}}%
    }}
    
\renewcommand\footnoterule{\textcolor{unibo}{\kern-3pt \hrule width 2in \kern 2.6pt}}


\renewcommand\thefootnote{\textcolor{unibo}{\arabic{footnote}}}

\usepackage{tocloft}
\renewcommand{\cftdot}{\textcolor{unibo}{.}}
\renewcommand{\cftchapaftersnum}{~\textcolor{unibo}{|}~}


\font\grassettogreco=cmmib10
\font\scriptgrassettogreco=cmmib7
\font\scriptscriptgrassettogreco=cmmib10 at 5 truept
\font\sansserif=cmss10
\font\scriptsansserif=cmss10 at 7 truept
\font\scriptscriptsansserif=cmss10 at 5 truept
\textfont13=\grassettogreco
\scriptfont13=\scriptgrassettogreco
\scriptscriptfont13=\scriptscriptgrassettogreco
\def\bgr{\fam=13}
\textfont14=\sansserif
\scriptfont14=\scriptsansserif
\scriptscriptfont14=\scriptscriptsansserif
\def\ssm{\fam=14}

\input amssym.def
\input amssym.tex
%------------------------------------------------------------------
%   definizione lettere latine: GRASSETTO
%------------------------------------------------------------------
\def \abf{{\bf a}}
\def \bbf{{\bf b}}
\def \cbf{{\bf c}}
\def \dbf{{\bf d}}
\def \ebf{{\bf e}}
\def \fbf{{\bf f}}
\def \gbf{{\bf g}}
\def \hbf{{\bf h}}
\def \jbf{{\bf j}}
\def \kbf{{\bf k}}
\def \ibf{{\bf i}}
\def \lbf{{\bf l}}
\def \mbf{{\bf m}}
\def \nbf{{\bf n}}
\def \obf{{\bf o}}
\def \pbf{{\bf p}}
\def \qbf{{\bf q}}
\def \rbf{{\bf r}}
\def \sbf{{\bf s}}
\def \tbf{{\bf t}}
\def \ubf{{\bf u}}
\def \vbf{{\bf v}}
\def \xbf{{\bf x}}
\def \ybf{{\bf y}}
\def \wbf{{\bf w}}
\def \zbf{{\bf z}}
%----------------
\def \Abf{{\bf A}}
\def \Bbf{{\bf B}}
\def \Cbf{{\bf C}}
\def \Dbf{{\bf D}}
\def \Ebf{{\bf E}}
\def \Fbf{{\bf F}}
\def \Gbf{{\bf G}}
\def \Hbf{{\bf H}}
\def \Jbf{{\bf J}}
\def \Kbf{{\bf K}}
\def \Ibf{{\bf I}}
\def \Lbf{{\bf L}}
\def \Mbf{{\bf M}}
\def \Nbf{{\bf N}}
\def \Obf{{\bf O}}
\def \Pbf{{\bf P}}
\def \Qbf{{\bf Q}}
\def \Rbf{{\bf R}}
\def \Sbf{{\bf S}}
\def \Tbf{{\bf T}}
\def \Ubf{{\bf U}}
\def \Vbf{{\bf V}}
\def \Xbf{{\bf X}}
\def \Ybf{{\bf Y}}
\def \Wbf{{\bf W}}
\def \Zbf{{\bf Z}}
%------------------------------------------------------------------
% Laine a bastoncello per gli operatori (DEFINIZIONI DI GS 28-11-1996)
%                                        NON  MODIFICARE !!!
%------------------------------------------------------------------
\font\sans=cmss10
%----------------
\def \Aop{{\mathchardef\alpha="710B \ssm \char'101}}
\def \Bop{{\mathchardef\alpha="710B \ssm \char'102}}
\def \Cop{{\mathchardef\alpha="710B \ssm \char'103}}
\def \Dop{{\mathchardef\alpha="710B \ssm \char'104}}
\def \Eop{{\mathchardef\alpha="710B \ssm \char'105}}
\def \Fop{{\mathchardef\alpha="710B \ssm \char'106}}
\def \Gop{{\mathchardef\alpha="710B \ssm \char'107}}
\def \Hop{{\mathchardef\alpha="710B \ssm \char'110}}
\def \Iop{{\mathchardef\alpha="710B \ssm \char'111}}
\def \Jop{{\mathchardef\alpha="710B \ssm \char'112}}
\def \Kop{{\mathchardef\alpha="710B \ssm \char'113}}
\def \Lop{{\mathchardef\alpha="710B \ssm \char'114}}
\def \Mop{{\mathchardef\alpha="710B \ssm \char'115}}
\def \Nop{{\mathchardef\alpha="710B \ssm \char'116}}
\def \Oop{{\mathchardef\alpha="710B \ssm \char'117}}
\def \Pop{{\mathchardef\alpha="710B \ssm \char'120}}
\def \Qop{{\mathchardef\alpha="710B \ssm \char'121}}
\def \Rop{{\mathchardef\alpha="710B \ssm \char'122}}
\def \Sop{{\mathchardef\alpha="710B \ssm \char'123}}
\def \Top{{\mathchardef\alpha="710B \ssm \char'124}}
\def \Uop{{\mathchardef\alpha="710B \ssm \char'125}}
\def \Vop{{\mathchardef\alpha="710B \ssm \char'126}}
\def \Wop{{\mathchardef\alpha="710B \ssm \char'127}}
\def \Xop{{\mathchardef\alpha="710B \ssm \char'130}}
\def \Yop{{\mathchardef\alpha="710B \ssm \char'131}}
\def \Zop{{\mathchardef\alpha="710B \ssm \char'132}}
%-----------------------------
\def \aop{{\mathchardef\alpha="710B \ssm \char'141}}
\def \bop{{\mathchardef\alpha="710B \ssm \char'142}}
\def \cop{{\mathchardef\alpha="710B \ssm \char'143}}
\def \dop{{\mathchardef\alpha="710B \ssm \char'144}}
\def \eop{{\mathchardef\alpha="710B \ssm \char'145}}
\def \fop{{\mathchardef\alpha="710B \ssm \char'146}}
\def \gop{{\mathchardef\alpha="710B \ssm \char'147}}
\def \hop{{\mathchardef\alpha="710B \ssm \char'150}}
\def \iop{{\mathchardef\alpha="710B \ssm \char'151}}
\def \jop{{\mathchardef\alpha="710B \ssm \char'152}}
\def \kop{{\mathchardef\alpha="710B \ssm \char'153}}
\def \lop{{\mathchardef\alpha="710B \ssm \char'154}}
\def \mop{{\mathchardef\alpha="710B \ssm \char'155}}
\def \nop{{\mathchardef\alpha="710B \ssm \char'156}}
\def \oop{{\mathchardef\alpha="710B \ssm \char'157}}
\def \pop{{\mathchardef\alpha="710B \ssm \char'160}}
\def \qop{{\mathchardef\alpha="710B \ssm \char'161}}
\def \rop{{\mathchardef\alpha="710B \ssm \char'162}}
\def \sop{{\mathchardef\alpha="710B \ssm \char'163}}
\def \top{{\mathchardef\alpha="710B \ssm \char'164}}
\def \uop{{\mathchardef\alpha="710B \ssm \char'165}}
\def \vop{{\mathchardef\alpha="710B \ssm \char'166}}
\def \wop{{\mathchardef\alpha="710B \ssm \char'167}}
\def \xop{{\mathchardef\alpha="710B \ssm \char'170}}
\def \yop{{\mathchardef\alpha="710B \ssm \char'171}}
\def \zop{{\mathchardef\alpha="710B \ssm \char'172}}
%------------------------------------------------------------------
%   definizione lettere greche: GRASSETTO (NUOVA DEFINIZIONE DI GS
%                                          28-11-1996; NON TOCCARE !!)
%------------------------------------------------------------------
\font\gr=cmmib10
\def\Gammabf{{\mathchardef\Gamma="7100 \bgr \Gamma}}
\def\Deltabf{{\mathchardef\Delta="7101 \bgr \Delta}}
\def\Thetabf{{\mathchardef\Theta="7102 \bgr \Theta}}
\def\Lambdabf{{\mathchardef\Lambda="7103 \bgr \Lambda}}
\def\Xibf{{\mathchardef\Xi="7104 \bgr \Xi}}
\def\Pibf{{\mathchardef\Pi="7105 \bgr \Pi}}
\def\Sigmabf{{\mathchardef\Sigma="7106 \bgr \Sigma}}
\def\Upsilonbf{{\mathchardef\Upsilon="7107 \bgr \Upsilon}}
\def\Phibf{{\mathchardef\Phi="7108 \bgr \Phi}}
\def\Psibf{{\mathchardef\Psi="7109 \bgr \Psi}}
\def\Omegabf{{\mathchardef\Omega="710A \bgr \Omega}}
\def\alphabf{{\mathchardef\alpha="710B \bgr \alpha}}
\def\betabf{{\mathchardef\beta="710C \bgr \beta}}
\def\gammabf{{\mathchardef\gamma="710D \bgr \gamma}}
\def\deltabf{{\mathchardef\delta="710E \bgr \delta}}
\def\epsilonbf{{\mathchardef\epsilon="710F \bgr \epsilon}}
\def\zetabf{{\mathchardef\zeta="7110 \bgr \zeta}}
\def\etabf{{\mathchardef\eta="7111 \bgr \eta}}
\def\thetabf{{\mathchardef\theta="7112 \bgr \theta}}
\def\iotabf{{\mathchardef\iota="7113 \bgr \iota}}
\def\kappabf{{\mathchardef\kappa="7114 \bgr \kappa}}
\def\lambdabf{{\mathchardef\lambda="7115 \bgr \lambda}}
\def\mubf{{\mathchardef\mu="7116 \bgr \mu}}
\def\nubf{{\mathchardef\nu="7117 \bgr \nu}}
\def\xibf{{\mathchardef\xi="7118 \bgr \xi}}
\def\pibf{{\mathchardef\pi="7119 \bgr \pi}}
\def\rhobf{{\mathchardef\rho="711A \bgr \rho}}
\def\sigmabf{{\mathchardef\sigma="711B \bgr \sigma}}
\def\taubf{{\mathchardef\tau="711C \bgr \tau}}
\def\upsilonbf{{\mathchardef\upsilon="711D \bgr \upsilon}}
\def\phibf{{\mathchardef\phi="711E \bgr \phi}}
\def\chibf{{\mathchardef\chi="711F \bgr \chi}}
\def\psibf{{\mathchardef\psi="7120 \bgr \psi}}
\def\omegabf{{\mathchardef\omega="7121 \bgr \omega}}
\def\varepsilonbf{{\mathchardef\varepsilon="7122 \bgr \varepsilon}}
\def\varthetabf{{\mathchardef\vartheta="7123 \bgr \vartheta}}
\def\varpibf{{\mathchardef\varpi="7124 \bgr \varpi}}
\def\varrhobf{{\mathchardef\varrho="7125 \bgr \varrho}}
\def\varsigmabf{{\mathchardef\varsigma="7126 \bgr \varsigma}}
\def\varphibf{{\mathchardef\varphi="7127 \bgr \varphi}}
 \def\varomegabf{\varpibf}
%----------------------------------------------------------------------
%    lettere greche grassetto minuscole
%----------------------------------------------------------------------
\def\taubfsm{\tau}
\def\Phibfsm{\Phi}
\def\phibfsm{\Phi}
\def\omegabfsm{\omega }
%-----------------------------------------------------------------------
%	definizione lettere maiuscole  CORSIVE
%-----------------------------------------------------------------------
\def \Acal{{\cal A}}
\def \Bcal{{\cal B}}
\def \Ccal{{\cal C}}
\def \Dcal{{\cal D}}
\def \Ecal{{\cal E}}
\def \Fcal{{\cal F}}
\def \Gcal{{\cal G}}
\def \Hcal{{\cal H}}
\def \Ical{{\cal I}}
\def \Jcal{{\cal J}}
\def \Kcal{{\cal K}}
\def \Lcal{{\cal L}}
\def \Mcal{{\cal M}}
\def \Ncal{{\cal N}}
\def \Ocal{{\cal O}}
\def \Pcal{{\cal P}}
\def \Qcal{{\cal Q}}
\def \Rcal{{\cal R}}
\def \Scal{{\cal S}}
\def \Tcal{{\cal T}}
\def \Ucal{{\cal U}}
\def \Vcal{{\cal V}}
\def \Wcal{{\cal W}}
\def \Xcal{{\cal X}}
\def \Ycal{{\cal Y}}
\def \Zcal{{\cal Z}}
%------------------------------------------------------------------
%   definizoni lettere greche abbreviate
%------------------------------------------------------------------
\def \alf{\alpha}
\def \bet{\beta}
\def \gam{\gamma}
\def \del{\delta}
\def \eps{\epsilon}
\def \kap{\kappa}
\def \thet{\theta}
\def \sig{\sigma}
\def \ome{\omega}
\def \Alf{\Alpha}
\def \Bet{\Beta}
\def \Gam{\Gamma}
\def \Del{\Delta}
\def \Eps{\Epsilon}
\def \The{\Theta}
\def \Kap{\Kappa}
\def \Sig{\Sigma}
\def \Ome{\Omega}
%
%--------------------------------------------------------------
%  Fonti vari
%--------------------------------------------------------------
%-------------- fonti grassetti per titoli -------------------
\font\titoletto=cmbx10 scaled \magstep  2
\font\titolino=cmbx10 scaled \magstep  3
\font\titolo=cmbx10 scaled \magstep 4
\font\titolone=cmbx10 scaled \magstep 5
%\font\titolissimo=cmbx10 at 30 pt
%--------------------------------------------------------------
%-----------------------fonti grandi grassetti 12 punti---------
\font \bfla=cmbx10
%-----------------------fonti piccoli grassetti 8 punti---------
\font \bfsm=cmbx8
%-----------------------fonti romani piccoli 8 punti -----------
\font \rmsm=cmr8
%-----------------------fonti romani piccolissimi 5 punti ------
\font \rmvsm=cmr5
%-----------------------fonti italici piccoli 7 punti -----------
\font \itsm=cmmi8
%-----------------------stile formule con simboli a 7 punti-----
\font \slsm=cmsl8
\font  \slsmed=cmsl9
%-----------------------fonti slanted piccoli 7 punti -----------
\font \slvsm=cmsl10 at 5 pt
%-----------------------fonti italici piccoli 7 punti -----------

\def \sysm{\scriptstyle}
%---------------------------------------------------------------
%
%------------------------------------------------------------------
%
%	  Spaziature per Capitoli, sezioni, paragrafi (Sakurai)
%
%      \sksec va prima e \sksecc va dopo il titolo di una sezione
%
%------------------------------------------------------------------
% \def \skcap{\vskip .2 truecm \noindent \hrule \vskip 1.5 truecm \noindent }
\def \skcap{\line{} \vskip  2.0  truecm \noindent }
\def \skcapp{\vskip 6.5  truecm \noindent }
\def \sksec {\vskip 1.5 truecm   \line{} \vskip .5 truecm \noindent}
\def \sksecm{\vfill\eject  \line{} \vskip -1.5 truecm \noindent}
\def \sksecc {\nobreak \vskip 1. cm \nobreak \noindent }
\def \skpar {\vskip 1. truecm \line{} \vskip .5 truecm \noindent }
\def \skparr{\nobreak \vskip .8 truecm \nobreak \noindent }
\def \miniskpar {  \vskip  .75 truecm  \noindent }
\def \miniskparr{\nobreak \vskip .5 truecm \nobreak \noindent }
\def \skteor {\vskip .7 truecm}
\def \skteorr {\nobreak \vskip .7 truecm \nobreak }
\def \skipfig{\vskip .2 truecm \noindent}
\def \spazio{\vskip .5	  truecm \noindent }
\def \spaziom{\vskip -.5  truecm \noindent }
\def \spa{\vskip    .3	  truecm \noindent }
\def \spam{\vskip   -.3	  truecm \noindent }
\def\spi{\vskip .15	 truecm  \noindent}
%
\def \pan {\par \noindent}
\def \nod {\noindent}
\def\segue{\qquad \Longrightarrow \qquad}
\def\im#1{\item{#1}}
\def\imm#1{\itemitem{#1}}
\def \parit{\par\item{\null}}
\def \panit{\par\noindent \item{\null}}
%----------------------------------------------------------------------
%  MACRO PER GESTIONE TESTO A COLORI IN FUNZIONE DELL'HOST
%
%  Si usa, ad esempio,
%      \Colori{\Silicon}   equivalente a \Colori{1}
%
%      oppure
%      \Colori{\PC}        equivalente a \Colori{2}
%
%
   \newcount\Silicon \global\Silicon=1
   \newcount\PC \global\PC=2
   \newcount\parziale
   \def\Colori#1{\global\parziale=#1
                \ifnum\parziale=\Silicon
                    \input colors
                    \gdef\Color##1{\Black{##1}}                    
                \else\ifnum\parziale=\PC
                    \input colordvi
                    \gdef\textRGB##1{\textColor{##1 0.}}
                    \gdef\GrayA##1{\textGray{##1}}
                    \gdef\GrayB##1{\textGray{##1}}
                    \gdef\GrayC##1{\textGray{##1}}
                    \gdef\GrayD##1{\textGray{##1}}
                    \gdef\GrayE##1{\textGray{##1}}
                    \gdef\GrayF##1{\textGray{##1}}
                    \gdef\GrayG##1{\textGray{##1}}
                    \gdef\GrayH##1{\textGray{##1}}\fi                   
                \fi}        
%----------------------------------------------------------------------
%
%----------------------------------------------------------------------
%
%		  Definizioni varie
%
%----------------------------------------------------------------------
%	 Bra, ket, prodotti scalari, elementi di matrice
% ---------------------------------------------------------------------
%
\def\mean#1{\langle \,#1\,\rangle}
\def \ket#1{|{#1}\rangle}
\def \bra#1{\langle{#1}|}
\def \ketseq#1#2{|{#1}^{({#2})}\rangle	}
\def \braseq#1#2{ \langle{#1}^{({#2})} |}
\def \scal#1#2{\langle{#1}|{#2}\rangle}
\def \trace#1{ \hbox{Tr}({#1})	}
\def  \matel#1#2#3{ \langle {#1}|{#2}|{#3} \rangle  }
%------------------------------------------------------------------------
%	 Parentesi varie
%------------------------------------------------------------------------
\def \parton#1{\left({{#1}}\right)}
\def \parqua#1{\left[{{#1}}\right]}
\def \pargra#1{\left\{{{#1}}\right\}}
\def \parbar#1{\left|{{#1}}\right|}
\def \parbbar#1{\left | { \left |{{#1}}\right| } \right | }
\def \parmean#1{\left\langle{{#1}}\right\rangle}
%
%-----------------------------------------------------------------------
%      Derivate normali e parziali
%-----------------------------------------------------------------------
\def \der#1#2{{d{#1}\over d{#2}}}
\def \dert#1{{d{#1}\over dt}}
\def \derpt#1{{\partial{#1}\over \partial t}}
\def \derp#1#2{{\partial{#1} \over \partial{#2} }}
\def \derpsec#1#2{{\partial^2{#1} \over \partial{ {#2} ^2} }}
\def \Dert {{d \over dt}}
\def \Derpt {{\partial \over \partial t}}
\def \Derp#1{{\partial \over \partial{#1}} }
\def \Der#1{{d \over d{#1}} }
\def \dersect#1{{d^2{#1} \over dt^2 }}
\def \disp{\displaystyle}
%----------------------------------------------------------------------
%	Per indici piccoli
%---------------------------------------------------------------------
%--7 punti ---
\font \rmsmm=cmr7
\def \hbix#1{\rmsmm \hbox{ {#1} } }
%--5 punti ---
\def \hbixsm#1{\rmvsm \hbox{ {#1} } }
%----------------------------------------------------------------------
%   Macro per mettere una scritta sotto una formula
%	si usa come \underbrace
%---------------------------------------------------------------------
\def \underwrite#1{\mathop{\vtop{\ialign {##\crcr
$\hfil\displaystyle {#1}\hfil$\crcr\noalign{\kern3pt\nointerlineskip}
\crcr\noalign{\kern3pt}}}} \limits}
%--------------------------------------------------------------------
%		   caratteri speciali
%--------------------------------------------------------------------
%--  Caratteri barrati --
%--------------------------------------------------------------------
\def\lambdabar{{\mathchar'26\mskip-9mu\lambda}}
\def\Lpbar{\hbox{{\it $\Lambda p$}\hskip-7.7pt/}}
\def\bar#1{{\mathchar'26\mskip-9mu#1}}
\def\pbar{\hbox{{\it p}\hskip-5pt/}}
\def\kbar{\hbox{{\it k}\hskip-3.5pt/}}
\def\dbar{\hbox{{\it $\delta$}\hskip-5pt/}}
\def\lbar{{\mathchar'26\mskip-9mu\lambda}}
\def\Abar{\hbox{{\it A}\hskip-4.5pt/}}
%-- R,Z,T doppia barra
\def\Poinc{\hbox{I\hskip-1.7pt P}}
\def\Roinc{\hbox{I\hskip-1.7pt R}}
\def\Toinc{\hbox{I\hskip-1.1pt T}}
\def\Zoinc{\hbox{/\hskip-1.1pt Z}}
%---------------------------------------
%
%  Caratteri matematici
%----------------------------------------
\def\Realism{\Roinc}
\def\Torosm{\bf T}
\def\Sferasm{\bf S}
\def\Reali{\Bbb R}
\def\Interi{\Bbb Z}
\def\Toro{\Bbb T}
\def\Razionali{\Bbb Q}
\def\Naturali{\Bbb N}
\def\Complessi{\Bbb C}
\def\Sfera{\Bbb S}
\def\Ppro{\Bbb P}
%-------------------------------------------------------------------
%    Miscellanea
%------------------------------------------------------------------
\def\sqr#1#2{{\vcenter{\hrule height.#2pt
     \hbox{\vrule width.#2pt height#1pt \hskip#1pt
       \vrule width.#2pt}
     \hrule height.#2pt}}}
\def\square{\mathchoice\sqr34\sqr34\sqr{2.1}3\sqr{1.5}3}
\def\DAL{\hbox{\raise.250ex \hbox{$\sqr7{10}\,$}}} % DALambertiano
\def\und{\underline}
\def\pr{\prime}
\def\ov{\overline}
\def\cen{\centerline}
\def\ndt{\noindent}
\def\segue{\qquad \Longrightarrow \qquad}
%------------------------------------------------------------------
%    Bibliografia
%------------------------------------------------------------------
\def\bib#1.#2/#3/#4/#5/#6/#7.{\frenchspacing\item{[#1]}#2:\ {\it ``#3''}
~--~#4\ $\underline{\bf #5}$,\ #6 (#7)}
%------------------------------------------------------------------
%    Diagramma
%------------------------------------------------------------------
\def\diagramma#1#2#3#4#5#6#7#8{
 \vbox to 2.5cm{
       \hbox to 3cm{\hfil ${#1}$ \hfil}
       \hbox to 3cm{ ${#2}$\rightarrowfill ${#3}$ }
\hbox to 3cm{${#4} \Biggl\uparrow\hfill\Biggr\uparrow {#5}$}
       \hbox to 3cm{ ${#6}$\rightarrowfill ${#7}$ }
       \hbox to 3cm{\hfil ${#8}$ \hfil}   }    }
%-------------------------------------------------------------------------
%    quella  che  segue  e` la macro che serve a mettere una cosa sopra
%    l'altra:   #2 va sopra #1
%-------------------------------------------------------------------------
\def\sopra#1#2{{\raise 0.8 ex
\hbox{$
{{\scriptstyle \,{#2}}	\atop \displaystyle{#1}}
$}}
}
%------------------------------------------------------------------
%    Macro per figure
%------------------------------------------------------------------
\def \figura#1#2#3{\midinsert \centerline {\bf{Fig.#1}}\vskip #2 truecm
\vskip .3 truecm \noindent {\rmsm {#3}}\endinsert \noindent }
\def \figurac#1#2#3{\midinsert \centerline {\bf{Fig.#1}}\vskip #2 truecm
\vskip .3 truecm \centerline{\rmsm {#3}}\endinsert \noindent }
\def \figurat#1#2#3{\topinsert \centerline {\bf{Fig.#1}}\vskip #2 truecm
\vskip .3 truecm \centerline{\rmsm {#3}}\endinsert \noindent }
\def \figuraclong#1#2#3{\midinsert \centerline {\bf{Fig.#1}}\vskip #2 truecm
\vskip .3 truecm \noindent {\rmsm {#3}}\endinsert \noindent }
\def \figuratlong#1#2#3{\topinsert \centerline {\bf{Fig.#1}}\vskip #2 truecm
\vskip .3 truecm \noindent {\rmsm {#3}}\endinsert \noindent }
%-------
%------------------------------------------------------------------
%    Macro per figure, con inserimento di codice post_script.
%------------------------------------------------------------------
%	   macro per elaborazione su PC o su MVX
\def\figuraps#1#2#3#4#5{
\par
\midinsert
\centerline{\bf #4}
\vbox to #3 truecm{
\vskip #3 truecm
\ifnum #1 = 0	% ---  siamo su PC
\special {ps: plotfile #2}
\else		% ---  siamo su MVX
\special {#2 0 0 moveto 16} \fi
}
\centerline{#5}
\endinsert}
%------------------------------------------------------------------
% macros per libro di Meccanica
%------------------------------------------------------------------
\def\v#1{{\bf #1}}
\def\dim{{d}}	     %	 gradi di liberta'             (n)
\def\Lag{L}	     %	 lagrangiana		       (\Lcal)
\def\Azione{A}       %    Azione=\int L dt
\def\ala{a}	     %	 coeffiecienti metrica in T_2  (g)
\def\alaop{\aop}     %	 operatore associato	       (\gop)
\def\rbfG{ \rbf_G{\rbf_{_{\hbox{\rmvsm G}}}}  } % posizione centro massa
\def\grad{\hbox{grad}\,}
\def\saltopagina{\vfill\eject \line{} \vskip 1.    truecm  \noindent }
\def\secpagina{ \line{} \vskip 0.5  truecm \noindent}
%-------------------------------------------------------------------
%    formule piccole
%-------------------------------------------------------------------
\def\mini#1{$\sysm{{#1}}$}
\def\minidis#1{$$\sysm{{#1}}$$}
%-------------------------------------------------------------------
% Testatina del libro di DINAMICA
%  ripristinare il \nopagenumbers
%-------------------------------------------------------------------
% \vsize= 10. truecm
% \hsize= 10. truecm
\font\cofon = cmr6
\font\cobfon = cmbx6
\font\copi = cmr9
\def\codlib{{\copi\copyright}{\cofon 88-08- }{\cobfon 9820}}
\def\riga{\vskip .1  truecm   \hrule \vskip .2	    truecm \noindent }
% \nopagenumbers    % ******* commentato il 8/7/98 ******
% \headline=
\def\HeadLinea#1#2{	\headline={\vbox to 0pt{\vss\noindent
{\ifnum \pageno=1  \hfill {\bf \folio}		 %  prima pagina
\else {\ifodd \pageno			   % pagina sinistra
{\noindent     \hfill  {\it     #2} \quad {\bf \folio}
 }\riga
\else				 %pagina destra
{\noindent {\bf\folio} \quad  {\it  #1} \hfill 	}
\riga
\fi } \fi }			}}
}
\def\testatina#1#2{	\headline={\vbox to 0pt{\vss\noindent
{\ifnum \pageno=1  \hfill {\bf \folio}		 %  prima pagina
\else {\ifodd \pageno			   % pagina sinistra
{\noindent  \codlib   \hfill  {\it     #2} \quad {\bf \folio}
 }\riga
\else				 %pagina destra
{\noindent {\bf\folio} \quad  {\it  #1} \hfill \codlib	}
\riga
\fi } \fi }			}}
}
%
%   Primo argomento di \testata   titolo capitolo
%   Secondo argomento numero e titolo paragrafo
%
\def\testatinacap#1#2#3{	\headline={\vbox to 0pt{\vss\noindent
{\ifnum \pageno=#3  \hfill {\bf \folio}		 %  prima pagina
\else {\ifodd \pageno			   % pagina sinistra
{\noindent  \codlib   \hfill  {\it     #2} \quad {\bf \folio}
 }\riga
\else				 %pagina destra
{\noindent {\bf\folio} \quad  {\it  #1} \hfill \codlib	}
\riga
\fi } \fi }			}}
}
%
% terzo argomento pagina inizio capitolo
%
%------------------
%  date 
%------------------
\def\oggi{\number\day\space\ifcase\month
   \or gennaio\or febbraio\or marzo\or aprile\or maggio\or giugno\or
   luglio\or agosto\or settembre\or ottobre\or novembre\or dicembre
   \fi\space\number\year}
\def\today{\number\day\space\ifcase\month
   \or January\or February\or March\or April\or May\or June\or
   July\or August\or September\or October\or November \or December
   \fi\space\number\year}   
%--------------------------------------------------
%  Colori per TEX
%--------------------------------------------------
% \def\Colore#1#2#3{\special{ps: #1 #2 #3 setrgbcolor}}
% \def\Rosso{\Colore{1}{0}{0} }
% \def\Verde{\Colore{0}{1}{0} }
% \def\Blu{\Colore{0}{0.65}{1} }
% \def\Celeste{\Colore{0}{1}{1} }
% \def\Nero{\Colore{0}{0}{0} }
% \def\Bianco{\Colore{1}{1}{1} }
% \def\Rossoscuro{\Colore{0.80}{0.00}{0.20} }
% \def\Verdescuro{\Colore{0.00}{0.80}{0.20} }
% \def\Bluscuro{\Colore{0.20}{0.00}{0.80} }
% \def\Viola{\Colore{0.5}{0.0}{0.5} }
%------------------------------
%   Nuovi
%-------------------------------
% \def\Lilla{\Colore{1}{0}{1} }
% \def\Bluchiaro{\Colore{0}{0.75}{1} }
% \def\Blumedio{\Colore{0}{0.5}{1} }
% \def\Arancio{\Colore{1}{0.5}{0} }
% \def\Verdechiaro{\Colore{.5}{1}{0} }
% \def\Marrone{\Colore{.6}{.4}{0.2} }
% \def\Giallo{\Colore{1}{1}{0} }
% \def\Gialloscuro{\Colore{.8}{.7}{0} }
% \def\Rossochiaro{\Colore{1}{0}{0.5} }
% \def\Rosa{\Colore{1}{0.5}{.5} }
% \def\Grigio{\Colore{0.5}{0.5}{0.5} }
%-----------------------------
%
%   Per cambiare colore  basta fare  \Rosso  ....... \Blu .......
%
%   e cosi di seguito senza parentesi
%
%-----------------------------
%%%%% FRAME.TEX file : it contains the following MACROS
%% \frame, \tframe  to put formulas in BOX!

%%%%%%%%%%%%%%%%%%  MACRO for \frame, \tframe (Ortolani, January 94)
% \frame{Thing}   : Puts a rule frame around Thing
% \tframe{Thing}  : Puts a tight rule frame around Thing
%
\def\frame#1{\ifmmode\dframe{#1}\else\leavevmode\lower 2.4 pt
    \hbox{\vrule\unskip\vbox{\hrule\kern 1.5 pt\hbox{\kern
    1.5 pt{#1}\kern 0.5 pt}\kern 2 pt\hrule}\unskip\vrule}\fi}
\def\tframe#1{\hbox{\vrule\unskip\vbox{\hrule\hbox{#1}\hrule}\unskip\vrule}}
\def\dframe#1{\hbox{\vrule\unskip$\vcenter{\hrule\kern 3 pt\hbox
    {\kern 3 pt$\displaystyle{#1}$\kern3pt}\kern 3 pt\hrule}$\vrule}}

%%%%%%%%%%%%%% END of FRAME.TEX Formulas in Box!
\def\redframe#1{\Rosso \frame{\Nero #1 \Rosso}\Nero}
\def\Redframe#1{\Rosso \frame{\Nero #1 \Rosso}\Nero}

\def\math#1{\Nero$#1$\Blut}
\def\mathmini#1{\Nero$\scriptstyle{#1}$\Blut}
\def\minimath#1{\Nero$\scriptstyle{#1}$\Blut}
%  FINE
%-------
