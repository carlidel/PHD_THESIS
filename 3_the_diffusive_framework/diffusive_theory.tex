
\chapter{The diffusive framework}\label{ch:the_diffusive_framework}




\section{From stochastic Hamiltonians to Fokker-Planck equations}

\subsection{Diffusion over a toroidal surface}
\label{subsec:diffusion-over-a-toroidal-surface}

We start by investigating the process of fast diffusion of a probability distribution over a toroidal surface. We are interested in this process as such situations happens in action-angle Hamiltonian systems for the angle variable.

Let us start by defining the following notation for the noise of a Wiener process and its integral:
\begin{equation}
	w(t) = \int_0^t \xi(s)\,\mathrm{d}s \,, \qquad w_1(t)=\int_0^t w(t)\,\mathrm{d}s \,,
\end{equation}
where $\xi(s)$ is a regular stationary stochastic Gaussian process, i.e.\ we assume that the realizations $\xi(t)$ are continuous, with a correlation function that reads
\begin{equation}
    \langle\xi(t)\xi(t+\tau)\rangle = \sigma^2 \Omega(\gamma\tau) \,,\qquad \Omega(\gamma\tau) \simeq e^{-\gamma|\tau|}\,.
\end{equation}

We can write \(w_1(t)\) in the form
\begin{equation}
	w_1(t) = \int_0^t \int_0^s \xi(u)\,\mathrm{d}u\,\mathrm{d}s' = \int_0^t \xi(u)\int_u^s\,\mathrm{d}s'\,\mathrm{d}u = \int_0^t (s'-u)\xi(u)\,\mathrm{d}u \,.
\end{equation}
This form is convenient if we want to present the processes \(w(t)\) and \(w_1(s)\) in the standard form 
\begin{equation}
	\int_0^t K(t,s)\xi(s)\,\mathrm{d}s \,
\end{equation}
where \(K(t,s)\) is referred to as the \textit{kernel} of the noise. We have that \(K=1\) for \(w(t)\) and \(K=t-s\) for \(w_1(t)\). This implies \(\sigma^2 = t\) for Wiener noise \(w(t)\) while instead, for its integral \(w_1(t)\), we have
\begin{equation}
	\sigma^2(t) = \int_0^t (t-u)^2\,\mathrm{d}u = \frac{t^3}{3} \,.
\end{equation}

Let us consider a simple example of angular diffusion. Let \(x\) be an angular variable defined over the interval \([0,1]\). We define then
\begin{equation}
	\begin{aligned}
		\dot{x} &= \omega + y \mod{1}\,, \\
		\dot{y} &= \epsilon\xi(t)	\,.
	\end{aligned}
	\label{eq:langevin_toro}
\end{equation}
A diffusive solution for an \(\hat{x}\) defined on all \(\mathbb{R}\) in the form \(\rho(\hat{x},t)\) can be converted as a solution valid for \(x\) defined on the torus \(\rho_T(x,t)\) by using the following periodization
\begin{equation}
	\rho_T(x,t) = \sum_{k=-\infty}^{+\infty} \rho(x+k,t) \quad x\in[0,1]\,,\quad k = 0,\pm 1, \pm 2, \ldots \,.
\end{equation}

The solution of~\eqref{eq:langevin_toro} on \(\mathbb{R}\) is given by
\begin{equation}
	y = y_0 + \epsilon w(t)\,, \qquad x = x_0 + (\omega + \epsilon y_0) t + \epsilon w_1(t) \,,
\end{equation}
by writing the deterministic part of \(x(t)\) in the form \(\langle x(t) \rangle = x_0 + (\omega + \epsilon y_0) t\), we can write the probability density in \(\mathbb{R}\) in the form
\begin{equation}
	\rho(x,t) = \frac{\exp{\frac{-(x-\langle x(t)\rangle)^2}{2\sigma^2}}}{\sqrt{2\pi\sigma^2}}\,, \qquad \sigma^2 = \epsilon^2 \frac{t^3}{3}\,,
\end{equation}
we can then apply the periodization and finally obtain
\begin{equation}
	\rho_T(x,t) = \sum_{n=-\infty}^{+\infty} \frac{\exp{\frac{-(x-\langle x(t)\rangle + n)^2}{2\sigma^2}}}{\sqrt{2\pi\sigma^2}} \,\qquad n = 0,\pm 1, \pm 2, \ldots\,.
\end{equation}

If we represent the solution in the form of Fourier series
\begin{equation}
	\rho_T(x,t) = \sum_{k=-\infty}^{+\infty} f_k(t)e^{2\pi i k x} \,,
\end{equation}
the \(f_k\) coefficients are given by
\begin{align}
	f_k &= \int_0^1 e^{-2\pi i k x} \rho_T(x,t)\,\mathrm{d}x \\
	&= \sum_{n=-\infty}^{+\infty} \frac{1}{\sqrt{2\pi\sigma^2}} \int_0^1 \exp{\frac{-(x-\langle x(t) \rangle + n)^2}{2\sigma^2} -2\pi i k x}\,\mathrm{d}x \,.
\end{align}
By assuming real \(f_k\) coefficients, we obtain
\begin{equation}
	\rho_T(x,t) = 1+2\sum_{k=1}^\infty e^{-2\pi^2\sigma^2k^2} f_k \cos(2\pi k (x-\langle x(t) \rangle)) \,.
\end{equation}

The most important aspect of this example, is how \(\rho_T\) relaxes to the uniform distribution as \(e^{-\sigma^2}\), where, in this scenario, we have
\begin{equation}
	\sigma^2 \propto t^{3} \,,
\end{equation}
while, instead, for the Wiener process that affects \(y\), we expect to have
\begin{equation}
	\sigma^2 \propto t \,.
\end{equation}

\subsubsection*{Application to a Hamiltonian system}

Let us consider the following perturbed Hamiltonian system
\begin{equation}
	\ham = \ham_0(I) - \epsilon \phi \xi(t)\,.
\end{equation}
From this Hamiltonian, we have the equations of motion
\begin{equation}
	\begin{aligned}
		{I'} &= \epsilon\xi\,, \\
		{\phi'} &= \omega(I)\,,
	\end{aligned}
\end{equation}
from which follows immediately
\begin{equation}
	I = I_0 + \epsilon w(t)\,.
\end{equation}
If we now execute a Taylor expansion for \({\phi'}\), we obtain
\begin{equation}
	{\phi'} = \omega(I_0) + \omega'(I_0)\epsilon w(t) + \mathcal{O}\epsilon^2)\,,
\end{equation}
which, if integrated, leads to
\begin{equation}
	\phi = \phi_0 + \omega(I_0)t + \epsilon \omega'(I_0) w_1(t)\,.
\end{equation}
This kind of solution, as seen above, implies the following variance for $\phi$
\begin{equation}
	\sigma_{\phi}^2 \propto t^3\,,
\end{equation}
while instead, for \(I\), the variance reads
\begin{equation}
	\sigma_{I}^2 \propto t\,.
\end{equation}

\subsection{Averaging a 1\textsc{d} stochastic Hamiltonian}
\label{ssc:averaging-1D}

Let us consider generic 1\textsc{d} stochastically perturbed Hamiltonian system in action-angle coordinates, with the origin $I=0$ in an elliptic point of \(\ham_0\). The Hamiltonian reads
\begin{equation}
	\ham = \ham_0(I) + \epsilon \xi(t) \ham_1(\phi, I) \,.
\end{equation}
We have that the probability density \(\rho(\phi, I; t)\) satisfies the equation
\begin{equation}
	\pdv{\rho}{t}\/ + \Omega \pdv{\rho}{\phi}\/ = \frac{\epsilon^2 \sigma^2}{2}\{\ham_1, \{\ham_1,\rho\}\}
	\label{eq:stoc_ham_complex_starting}
\end{equation}
where \(\{\ ,\}\) are the Poisson brackets and \(\Omega(I) = \partial \ham_0/\partial I\).

From the results presented above, we know that a toroidal angle variable relaxes in much faster times compared to the action relaxation scales. This allows us to simplify this last equation into a Fokker-Planck equation for \(t\gg t_{\phi}\). The resulting equation, describing the distribution evolution for the action variable \(\rho(I,t)\), averaged over the angle variable, will then read
\begin{equation}
	\pdv{}{t}\/\rho(I,t) = \pdv{}{I}\/D(I)\pdv{}{I}\/\rho(I,t) \,,
\end{equation}
where
\begin{equation}
	D(I) = \frac{\epsilon^2 \sigma^2}{2}\frac{1}{2\pi}\int_0^{2\pi}\left(\pdv{\ham_1}{\phi}\/\right)^2 \,d\phi \equiv \frac{\epsilon^2 \sigma^2}{2} \left\langle \left( \pdv{\ham_1}{\phi}\/  \right)^2 \right\rangle_\phi \,.
\end{equation}

\subsubsection*{Proof}
Let us start from the following stochastic Liouville equation
\begin{equation}
	\pdv{\rho}{t}\/ + \Omega(I) \pdv{\rho}{\phi}\/ + \epsilon\xi(t)\{\rho,\ham_1\} = 0\,,
	\label{eq:dimo_liouville}
\end{equation}
from here, we can write \(\rho\) in the form \(\rho = \rho_0 + \epsilon\rho_1\), where \(\rho_0\) is the average component of the distribution and \(\rho_1\) the fluctuating component with zero mean. Considering that \(\langle\xi\rangle = 0\), we have that the average value of Eq.~\eqref{eq:dimo_liouville} can be written as
\begin{equation}
	\pdv{\rho_0}{t}\/ + \Omega(I)\pdv{\rho_0}{\phi}\/ + \epsilon^2\{\langle\xi(t)\rho_1\rangle, \ham_1\} = 0,.
	\label{eq:dimo_media}
\end{equation}
If we now subtract~\eqref{eq:dimo_media} from~\eqref{eq:dimo_liouville}, we obtain the equation
\begin{equation}
	%TODO: check this passages again (never trust past Carlo too much...)
	\pdv{\rho_1}{t}\/ + \Omega(I) \pdv{\rho_1}{\phi}\/ = -\xi(t)\{\rho_0, \ham_1\} + O(\epsilon)\,
\end{equation}
which we now want to solve for \(\rho_1\), and substitute it into Eq.~\eqref{eq:dimo_media}.

To archive this, let us consider the following change of variables:
\begin{equation}
	\begin{aligned}
		\phi &\to \phi - \Omega \tau\,, \\
		t &\to t - \tau\,.
	\end{aligned}		
\end{equation}
This change of variables allows us to group the last equation in the form
\begin{equation}
	\dv{}{\tau}\/\rho_1(\phi - \Omega \tau, I; t - \tau) = \xi(t- \tau)\{\rho_0, \ham_1\}(\phi - \Omega \tau, I; t - \tau)\,.
\end{equation}
If we now integrate this last equation from \(\tau = 0\) to \(\tau = t\), we reach the following expression
\begin{equation}
	\rho_1(\phi, I, t) = -\int_0^t\{\rho_0, \ham_1\}(\phi - \Omega \tau, I, t - \tau)\xi(t-\tau)\,\mathrm{d}\tau\,,
\end{equation}
where we took advantage of the fact that \(\rho_1(\phi, I; 0)=0\). If we now multiply both members by \(\xi(t)\), and compute the average over all the noise realizations, we have, for the case of a Wiener noise:
\begin{equation}
	\langle \xi(t)\rho_1(\phi, I; t) \rangle = -\frac{1}{2}\sigma^2 \{\rho_0, \ham_1\}(\phi, I; t)\,.
\end{equation}

Finally, if we insert this last result into Eq.~\eqref{eq:dimo_media}, we obtain, neglecting terms of order \(\epsilon^3\) or higher, the equation for the average density
\begin{equation}
 	\pdv{\rho_0}{t}\/ + \Omega (I) \pdv{\rho_0}{\phi}\/ = \frac{\epsilon^2 \sigma^2}{2}\{\{\rho_0, \ham_0\}, \ham_0\}.
\end{equation}

If we have that the relaxation timescale of the angle is faster than the action diffusion timescale, we can say, with good approximation, that \(\rho=\rho(I,t)\) does not depend on the angle variable \(\phi\). With this motivation, we can expand the double Poisson bracket expression, which reads
\begin{equation}
	\{\{\rho_0, \ham_0\}, \ham_0\} = \pdv{}{I}\/\left[\left(\pdv{\ham_1}{\phi}\/\right)^2 \pdv{\rho_0}{I}\/\right] - \pdv{}{\phi}\/\left[\pdv{\rho_0}{I}\/\pdv{\ham_1}{\phi}\/\pdv{\ham_1}{I}\/\right]\,,
\end{equation}
and take the angular average of this last equation. The second term in the r.h.s\ will then become zero, while the first term constitutes the expression of \(D(I)\), which now has to be integrated over the whole torus.

\subsection{Averaging principle for generic stochastic Hamiltonians}

We now generalize this last result to stochastically perturbed Hamiltonian systems at higher dimensions. Let us start by considering a generic Hamiltonian system
\begin{equation}
	\ham = \ham_0(I) + \xi(t)\ham_1(\phi, I)\,,
\end{equation}
where \((\phi, I)\) are now, respectively, variables in $\mathbb{T}^n$ and $\mathbb{R}^n$ with $n>1$, and the noise realization \(\xi(t)\) depends on the initial condition of the orbit.

We now introduce the slow variable
\begin{equation}
	\phi = \phi - \Omega(I)t\,,
\end{equation}
where \(\Omega(I) = \frac{\partial \ham_0}{\partial I}(I)\). This leads us the new Hamiltonian
\begin{equation}
	\ham = \xi(t)\ham_1(\phi+\Omega(I)t,I)\,,
\end{equation}
via the generating function
\begin{equation}
	F(\phi,I) = \phi I - \ham_0(I)t\,.
\end{equation}

To find an approximate solution of the stochastic dynamic, we can consider the evolution of the angle-action variables for a time \(T\gg \lambda\). From the new Hamiltonian we now have
\begin{equation}
	\begin{aligned}
		\Delta\phi_j(T) &= \int_0^T\pdv{\ham_1}{I_j}\/(\phi+\Omega(I)t, I)\xi(t)\,\mathrm{d}t -\\
		&\qquad - \int_0^T t\pdv{\ham_1}{\phi_k}\/(\phi + \Omega(I)t,I)\pdv{\Omega_k}{I_j}\xi(t)\,\mathrm{d}t\\ \,,
		\Delta I_j(T) &= -\int_0^T\pdv{\ham_1}{\phi_j}\/(\phi+\Omega(I)t,I)\xi(t)\,\mathrm{d}t\,,
	\end{aligned}
	\label{eq:averaging-generic-hams}
\end{equation}
where \(\Delta \phi_j(T)=\phi(T)-\phi(0)\) and \(\Delta I(T)=I(T)-I(0)\).

We now focus our attention on the second integral in the expression of \(\Delta\phi_j(T)\). By performing an integration by parts, where we also truncate the expansion to higher orders, we reach the expression
\begin{multline}
	\int_0^T t\pdv{\ham_1}{\phi_k}\/(\phi+\Omega(I)t,I)\pdv{\Omega_k}{I_j}\xi(t)\,\mathrm{d}t \simeq \\
	\simeq \pdv{\Omega_k}{I_j}\/\bigg[T\int_0^T\pdv{\ham_1}{\phi_k}\/(\phi+\Omega(I)t,I)\xi(t)\,\mathrm{d}t\, - \\
	- \int_0^T\int_0^t\pdv{\ham_1}{\phi_j}\/(\phi+\Omega(I)s, I)\xi(s)\,\mathrm{d}s\,\mathrm{d}t\bigg]\,,
\end{multline}
from here, we can combine the resulting two integrals together and replace their arguments using the expression for $\Delta I_j(T)$ given in Eq.~\eqref{eq:averaging-generic-hams}, and obtain
\begin{equation}
	\int_0^T t\pdv{\ham_1}{\phi_k}\/(\phi+\Omega(I)t,I)\pdv{\Omega_k}{I_j}\xi(t)\,\mathrm{d}t \simeq \pdv{\Omega_k}{I_j}\int_0^T[\Delta I_k(T)-\Delta I_k(t)]\,\mathrm{d}t\,.
\end{equation}
This implies that, if the action dynamics can be considered a stationary process, the main contribution to the angular dynamics is given by
\begin{equation}
	\Delta \phi_j \simeq -\pdv{\Omega_k}{I_j}\int_0^T\Delta I_k(t)\,\mathrm{d}t\,.
\end{equation}

If the system is non-degenerate, % i.e.\ the matrix \(\partial\Omega_k/\partial I_j\) is not singular,
the increment of $\phi$ are the integral of the increments of $I$ and, as illustrated above in this Chapter, we expect consequently a faster relaxation to a uniform distribution. From these considerations we have that, when considering the evolution of $I$, we can approximate the $\phi$ distribution with a uniform one. If we now expand the expression of $\Delta I_j$ up to terms of order \(\mathcal{O}(\norm{\ham_1}^2)\), we obtain the following:
\begin{multline}
	\Delta I_j = \int_0^T \pdv{\ham_1}{\phi_j}\xi(s)\,\mathrm{d}s + \int_0^T\int_0^t \pdv{\ham_1}{I_k}{\phi_j}\pdv{\ham_1}{\phi_k}\xi(t)\xi(s)\,\mathrm{d}s\,\mathrm{d}t - \\
	- \int_0^T\int_0^t\pdv{\ham_1}{\phi_k}{\phi_j}\left[\pdv{\ham_1}{I_k}-\pdv{\Omega}{I_k}\pdv{\ham_1}{\phi_k}t\right]\xi(t)\xi(s)\,\mathrm{d}s\,\mathrm{d}t\,.
\end{multline}

If then we assume that the distribution of \(\phi\) is uniform, we can average on both the noise realizations and on $\phi$. We then can consider the average of the stochastic dynamics of \(I\), which now reads
\begin{equation}
	\begin{aligned}
		\langle\Delta I_j\rangle_\phi &= \pdv{}{I_k}\int_0^T\int_0^t \left\langle\pdv{\ham_1}{\phi_j}\pdv{\ham_1}{\phi_k}\right\rangle e^{-(t-s)/\lambda}\,\mathrm{d}s\,\mathrm{d}t\\
		&= \frac{1}{2}\pdv{}{I_k}\left\langle\pdv{\ham_1}{\phi_j}\pdv{\ham_1}{\phi_k}\right\rangle\int_0^T\int_0^T e^{-\abs{t-s}/\lambda}\,\mathrm{d}s\,\mathrm{d}t\,,
	\end{aligned}
\end{equation}
where we have neglected the terms that are pure derivatives of $\phi$, and considered the mean and decorrelation properties of \(\xi(t)\). The corresponding variance is then estimated by
\begin{multline}
	\left\langle(\Delta I_j - \langle\Delta I_j\rangle_\phi)(\Delta I_k - \langle\Delta I_k\rangle_\phi)\right\rangle_\phi =\\
	= \left\langle\pdv{\ham_1}{\phi_j}\pdv{\ham_1}{\phi_k}\right\rangle\int_0^T\int_0^T e^{-\abs{t-s}/\lambda}\,\mathrm{d}s\,\mathrm{d}t\,.
\end{multline}

We can then describe the action dynamics as a stochastic process of the form
\begin{equation}
	\Delta I_j = -\sqrt{T\lambda\norm{\ham_1}^2}\sqrt{\left\langle\pdv{\hat\ham_1}{\phi_j}\pdv{\hat\ham_1}{\phi_l}\right\rangle}\,\hat{\xi}_l + \frac{T\lambda\norm{\ham_1}^2}{2}\pdv{}{I_k}\left\langle\pdv{\hat\ham_1}{\phi_j}\pdv{\hat\ham_1}{\phi_k}\right\rangle\,
	\label{eq:generic-ham-average-almost}
\end{equation}
where \(\hat{\ham_1}=\ham_1/\norm{\ham_1}\) and \(\hat{\xi}_l\) are identical independent distributed random variables with zero mean value and unitary variance. We can interpret the quantity \(T\lambda\norm{\ham_1}^2\) as the time step \(\Delta\tau\) of the diffusion time, we highlight however that it has a dimension of an action. Consequently, we then require that \(T\) should be sufficiently long in order to consider the angles relaxing to a uniform distribution and the noise decorrelated, but \(\norm{\ham_1}^2\) has to be so small that the actions do not evolve in a time \(T\). 

If these conditions are maintained, introducing the notation \(\tau = \lambda\norm{\ham_1}^2 t\), we have that Eq.~\eqref{eq:generic-ham-average-almost} is the approximation of the solution of the stochastic differential equation
\begin{equation}
	\mathrm{d}I_j = -\sqrt{\left\langle\pdv{\hat\ham_1}{\phi_j}\pdv{\hat\ham_1}{\phi_l}\right\rangle}\,\mathrm{d}\omega_l(\tau) + \frac{1}{2}\pdv{}{I_k}\left\langle\pdv{\hat\ham_1}{\phi_j}\pdv{\hat\ham_1}{\phi_k}\right\rangle\,\mathrm{d}\tau\,.
	\label{eq:final-ham-approximation}
\end{equation}

% In order to completely check the consistency of our claims, let us consider again the angle dynamics in the diffusion time (we omit for convenience the indices)
% \begin{equation}
% 	\Delta \phi = \left\langle\pdv{\hat{\ham}_1}{I}\right\rangle_{\phi}\sqrt{T\lambda\norm{\ham_1}^2}\,\xi -\pdv{\Omega}{I}\int_0^T(I(T)-I(t))\,dt
% \end{equation}
% by applying the approximation~\eqref{eq:final-ham-approximation} we just obtained, we can compute directly the fluctuating part of the action term \((I(T)-I(t))\) and obtain
% \begin{equation}
% 	\operatorname{Var}[I(T)-I(t)]=\left\langle\left(\pdv{\hat{\ham}_1}{\phi}\right)^2\right\rangle(\lambda\norm{\ham_1}^2)\frac{T^3}{2}
% \end{equation}
% At this point, if \(\partial\Omega/\partial I\propto \mathcal{O}(1)\) the stochastic approximation of the action variables we performed implies that the relaxation process for the angles \(\phi\) must be archived after a time \(t_\phi\), where \(\lambda\norm{\ham_1}^2 t_\phi^3 \simeq \mathcal{O}(1)\). Then, we can estimate
% \begin{equation}
% 	t_\phi \simeq \lambda^{-1/3}\norm{\ham_1}^{-2/3}
% \end{equation}
% so that the choice \(T\simeq t_\phi\) provides the result
% \begin{equation}
% 	\Delta \tau \simeq \lambda^{2/3}\norm{\ham_1}^{4/3}
% \end{equation}
% which vastly proves the assumption of fast angle relaxation, when compared to the diffusion timescale.

% The assumption on the fast relaxation of the angles \(\phi\), necessary to derive the equation~\eqref{eq:final-ham-approximation}, can be satisfied if the estimate \(\norm{\ham_1}^2 \ll \lambda^{-1}\) holds for the decorrelation timescale of the random fluctuations. This condition is necessary to describe the stochastic Hamiltonian dynamics as a diffusion process and it implies that the approach is consistent even if the Ljapounov exponent, characterizing the chaotic region, is small.

From this, we have that the evolution of the distribution function \(\rho(I,\tau)\) at the diffusion timescale is well approximated by the solution of the Fokker-Planck equation
\begin{equation}
	\pdv{\rho}{\tau}=\frac{1}{2}\pdv{}{I_j}\left\langle\pdv{\hat\ham_1}{\phi_j}\pdv{\hat\ham_1}{\phi_k}\right\rangle\pdv{}{I_k}\rho(I,\tau)
\end{equation}
where in this equation the slow diffusion time coefficient \(\tau\) has the dimension of the square of an action, so that the diffusion coefficient is adimensional.


\section{Functional form for $D(I)$}

We have reached the result that, under certain conditions, the evolution of an action distribution function $\rho_0(I, t)$ can be described by a Fokker-Planck equation in the form
\begin{equation}
    \pdv{\rho}{\tau}=\frac{1}{2}\pdv{}{I_j} D_{j,k}(I) \pdv{}{I_k}\rho(I,\tau)\,,
    \label{eq:final-fp}
\end{equation} 
where the diffusion coefficient $D_{j,k}(I)$ reads
\begin{equation}
    D_{j,k}(I) = \left\langle\pdv{\hat\ham_1}{\phi_j}\pdv{\hat\ham_1}{\phi_k}\right\rangle \,.
\end{equation}

In the context of accelerator physics, we have seen how the betatron motion can be described in terms of two action variables $(I_x, I_y)$, which define the two non-linear invariants for the two separate transverse planes. However, if we make the assumption that such a diffusive process takes place mainly along a one-dimensional direction, we can justify a 1\textsc{d} approach and reduce Eq.~\eqref{eq:final-fp} to the one degree of freedom case.

With the generic Hamiltonian notation, the resulting Fokker-Planck equation reads
\begin{equation}
    \pdv{\rho}{\tau}=\frac{1}{2}\pdv{}{I_j} D(I) \pdv{}{I_k}\rho(I,\tau)\,,
\end{equation}
with the diffusion coefficient given by
\begin{equation}
    D(I) = \frac{1}{\norm{\ham_1}^2} \left\langle \left( \pdv{\hat\ham_1}{\phi} \right)^2 \right\rangle_\phi \,.
\end{equation}

If we want to work with regular time units $t$, we can consider this form of Fokker-Plank equation:
\begin{equation}
    \pdv{\rho}{t}=\frac{\epsilon^2}{2}\pdv{}{I_j} D(I) \pdv{}{I_k}\rho(I,t)\,,
\end{equation}
where now $\epsilon^2 = \lambda\norm{\ham_1}^2$ represents the timescale constant of the diffusive process.

We are now interested in having an effective generic functional form for $D(I)$. To archive this, we follow the construction presented in~\cite{}, and references therein.

To estimate the norm of $\ham_1$, we have that perturbation theory suggests a possible estimate based on the asymptotic character of the perturbation series. Assuming that there are no low-order resonances in the phase-space, we can rely on the progressive expansion of the perturbative series, which gives us the following generic estimate~\cite{}:
\begin{equation}
    \norm{R_n(I)} \propto (n!)^\kappa \left(\frac{I}{I_\ast}\right)^{n/2}\,,
\end{equation}
where the factorial term takes into account the number of contributions due to the structure of the functional equations defining the perturbation series, the exponent $\kappa$ is related to the number of degrees of freedom, and the parameter $I_\ast$ represents an apparent radius of convergence of the perturbative series, and corresponds to the amplitude above which fast escape to infinity occurs.

For each $I$, we then have an optimal order for the Normal Form remainder defined by the relations
\begin{equation}
    n^\kappa=\left(\frac{I}{I_\ast}\right)^{1 / 2} \quad \Rightarrow \quad n=\left(\frac{I_\ast}{I}\right)^{1 / 2 \kappa} \,.
\end{equation}
If we then substitute this relation in the previous equation, we obtain a Nekhoroshev-like estimate that reads
\begin{equation}
    \left\|R_{\text {opt }}(I)\right\| \propto \exp \left[-\kappa\left(\frac{I_*}{I}\right)^{1 / 2 \kappa}\right] \,.
\end{equation}
This relation shows how the optimal estimate scales as a function of the action $I$. 

From this optimal remainder for the perturbation series, which can also be used as a measure of the long-term stability of the orbits at specific amplitudes, we can finally assume the following estimate for $\ham_1$:
\begin{equation}
    \left\|\ham_1(\phi, I)\right\| \simeq \exp \left[-\left(\frac{I_\ast}{I}\right)^{1/2\kappa}\right] \,.
\end{equation}
This estimate, enables us to define the following functional form for the diffusion coefficient
\begin{equation}
    D(I) = c \exp\left[-2\left(\frac{I_\ast}{I}\right)^{1/2\kappa}\right]\,,
    \label{eq:diffusion}
\end{equation}
where $c$ is a normalisation constant evaluated according to
\begin{equation}
    c \int_0^{I_\text{a}} D(I)\,\mathrm{d}I = 1 \,,
\end{equation}
so that $D(I)$ is normalised over the boundaries considered for the Fokker-Planck system, as $I_\text{a}$ represents the absorbing boundary condition, i.e.\ a barrier over which we can consider a particle as ``lost''. With this notation, the timescale of the process is entireley determined by $\epsilon^2$, however, variation of this notation can be freely considered.

Let us now consider this last form of Fokker-Planck equation, along with an absorbing boundary condition at $I_{\mathrm{a}}$, i.e.\ the phase-space limit beyond which an initial condition is considered lost. Note that $D(I)$ and $\rho$ have dimensions $[I^2t^{-1}]$ and $[I^{-1}]$, respectively.

In Fig.~\ref{fig:1} (top and centre), we consider the behaviour of $D(I)$ from Eq.~\eqref{eq:diffusion} for some values of $\kappa$ and for $\epsilon^2 = 1$. We can distinguish three regions for this type of function: $(i)$ a stable-core region for $I \ll I_\ast$, for which $D(I)$ has values decreasing to zero exponentially fast; $(ii)$ a ramp-up region for $I \lesssim I_\ast$, where $D(I)$ starts to have non-negligible values and changes from an exponential growth to an almost linear one; $(iii)$ a region for $I >  I_\ast$, where $D(I)$ features an almost linear growth (in logarithmic scale a saturation appears). These three regions are more or less distinguishable depending on the value of $\kappa$. 

In Fig.~\ref{fig:1} (bottom) we also display the result of the numerical integration of Eq.~\eqref{eq:fp} for some values of $\kappa$, performed on an initial distribution $\rho_0(I)=1$ with a Crank-Nicolson scheme~\cite{crank1947practical} (see Appendix~\ref{app_sec:numerical_integration_with_crank_nicolson} for some detail on the integration scheme used in our studies).
%, using $\varepsilon^2 = 1$ in all numerical simulations performed, and setting the absorbing boundary condition at $I/I_\ast=1.5$. 
It is also possible to observe in the shape of the distribution function a stable-core region, corresponding to $I/I_\ast \ll 1$, where $D(I)$ starts having values very close to zero, a fast decrease region and, finally, a saturation region, for $I/I_\ast = 1$ and beyond, where $\rho(I, t)$ assumes small values.

\begin{figure*}[htp]
    \centering
    \includegraphics[width=\textwidth]{4_probing_the_diffusive_behavior/figs/diffusion_coefficient.pdf}
    \caption{Top and centre: plot of $D(I)$ both in linear and logarithmic scale for three values of $\kappa$ as a function of $I/I_\ast$.
    %For the cases shown here, the choice $c=1$ has been made, to have comparable curves.
    Bottom: evolution of an uniform distribution (in grey) over the same time interval corresponding to $(t=10.0 \, [\text{a. u.}])$ for three values of $\kappa$. (Simulations parameter: $(I_\ast = 1.0\,[\sigma^2])$).}
    \label{fig:1}
\end{figure*}

\section{State of the art in diffusion measurements}

The concept of using a diffusive model to describe the behavior of beam halo and characterise the transverse loss rate is not a novel concept, as archiving a valid modeling of such phenomenon can lead to a better characterisation of long term losses and beam-halo formation.

A rather broad literature exist on the topic of diffusive models applied on accelerator machines, see, e.g.\ Refs.~{\cite{Burnod:205343,Meddahi:223301,PhysRevLett.68.33,gerasimov1992applicability,zimmermann1994transverse,PhysRevLett.77.1051,PhysRevSTAB.5.074001,flilleriii:pac03-rpag004} and references therein. %TODO:expand

An interesting line of research, which provided a valid approach for gathering local information about the value of the diffusion coefficient, is given by the usage of collimator scans presented in~\cite{MESS1994279}. Collimator scans can be used to probe the beam-halo dynamics and, in particular, to reconstruct the behaviour of the diffusion coefficient as a function of transverse amplitude~\cite{stancari2011diffusion,PhysRevSTAB.16.021003,PhysRevAccelBeams.23.044802}. The method of collimator scans has been intensively used at the LHC: it is based on small displacements of the jaws combined with the measurement of the beam losses. The displacements can be either inward or outward, and depending on the direction, the local losses feature different behaviour. The inspection of such features at different amplitudes finally provides local information about the diffusion coefficient.

The latest diffusion measurements, based on~\cite{stancari2011diffusion} diffusive framework, were taken during LHC Run2~\cite{Valentino:2280928}, and provided a consistent amound of collimator scan data which provided an overview of the halo diffusive phenomenology~\cite{PhysRevAccelBeams.23.044802}.  

Finally, the framework on which this work is based on has been developed and proposed in~\cite{Bazzani:2019lse,bazzani2020diffusion}, where the long-term behaviour of the beam dynamics and particle losses in circular accelerators is described by means of a global diffusive model. In this framework, the evolution of the beam distribution is described by means of a Fokker-Planck equation, in which the diffusion coefficient represents the key quantity to describe the beam dynamics. An excellent agreement between the data collected and the predictions from the FP model, where the approach consists in fitting the values of the model parameters $I_\ast$ and $\kappa$ to the collected data, has been observed.