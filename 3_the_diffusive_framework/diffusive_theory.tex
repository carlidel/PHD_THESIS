
\chapter{The diffusive framework}\label{ch:the_diffusive_framework}




\section{From stochastic Hamiltonians to Fokker-Planck equations}

\subsection{Diffusion over a toroidal surface}
\label{subsec:diffusion-over-a-toroidal-surface}

We shall now investigate the process of fast diffusion of a probability distribution over a toroidal surface. We are interested in this process as such situations happens in action-angle Hamiltonian systems for the angle variable.

We start by defining the following notation for the noise of a Wiener process and its integral:
\begin{equation}
	w(t) = \int_0^t \xi(s)\,ds \quad\quad w_1(t)=\int_0^t w(t)\,ds
\end{equation}
we can write \(w_1(t)\) in the form
\begin{equation}
	w_1(t) = \int_0^t \int_0^s \xi(u)\,du\,ds' = \int_0^t \xi(u)\int_u^s\,ds'\,du = \int_0^t (s'-u)\xi(u)\,du
\end{equation}
This can be convenient if we want to display the processes \(w(t)\) and \(w_1(s)\) in the standard form 
\begin{equation}
	\int_0^t K(t,s)\xi(s)\,ds
\end{equation}
where in this context \(K(t,s)\) is called the \textbf{kernel} of the noise. We have that \(K=1\) for \(w(t)\) and \(K=t-s\) for \(w_1(t)\). This implies \(\sigma^2 = t\) for Wiener noise \(w(t)\) while instead, for its integral \(w_1(t)\), we have
\begin{equation}
	\sigma^2(t) = \int_0^t (t-u)^2\,du = \frac{t^3}{3}
\end{equation}

Let us consider now a basic example of angular diffusion. Let \(x\) be an angular variable defined over the interval \([0,1]\). Let us define then
\begin{equation}
	\begin{aligned}
		\dot{x} &= \omega + y \mod{1} \\
		\dot{y} &= \epsilon\xi(t)	
	\end{aligned}
	\label{eq:langevin_toro}
\end{equation}
A diffusive solution for an \(\hat{x}\) defined on all \(\mathbb{R}\) in the form \(\rho(\hat{x},t)\) can be converted as a solution valid for \(x\) defined on the torus \(\rho_T(x,t)\) by using the following periodicization
\begin{equation}
	\rho_T(x,t) = \sum_{k=-\infty}^{+\infty} \rho(x+k,t) \quad x\in[0,1],\,k = 0,\pm 1, \pm 2, \ldots
\end{equation}

The solution of~\eqref{eq:langevin_toro} on \(\mathbb{R}\) is given by
\begin{equation}
	y = y_0 + \epsilon w(t) \qquad x = x_0 + (\omega + \epsilon y_0) t + \epsilon w_1(t)
\end{equation}
by writing the deterministic part of \(x(t)\) in the form \(\langle x(t) \rangle = x_0 + (\omega + \epsilon y_0) t\), we can write the probability density in \(\mathbb{R}\) in the form
\begin{equation}
	\rho(x,t) = \frac{\exp{\frac{-(x-\langle x(t)\rangle)^2}{2\sigma^2}}}{\sqrt{2\pi\sigma^2}} \qquad \sigma^2 = \epsilon^2 \frac{t^3}{3}
\end{equation}
we can apply the periodization and then obtain
\begin{equation}
	\rho_T(x,t) = \sum_{n=-\infty}^{+\infty} \frac{\exp{\frac{-(x-\langle x(t)\rangle + n)^2}{2\sigma^2}}}{\sqrt{2\pi\sigma^2}} \qquad n = 0,\pm 1, \pm 2, \ldots
\end{equation}

If we then represent the solution in the form of Fourier series
\begin{equation}
	\rho_T(x,t) = \sum_{k=-\infty}^{+\infty} f_k(t)e^{2\pi i k x}
\end{equation}
the \(f_k\) coefficients are given by
\begin{align}
	f_k &= \int_0^1 e^{-2\pi i k x} \rho_T(x,t)\,dx \\
	&= \sum_{n=-\infty}^{+\infty} \frac{1}{\sqrt{2\pi\sigma^2}} \int_0^1 \exp{\frac{-(x-\langle x(t) \rangle + n)^2}{2\sigma^2} -2\pi i k x}\,dx
\end{align}
by assuming real \(f_k\) coefficients, we obtain
\begin{equation}
	\rho_T(x,t) = 1+2\sum_{k=1}^\infty e^{-2\pi^2\sigma^2k^2} f_k \cos(2\pi k (x-\langle x(t) \rangle))
\end{equation}

The most important consequence of this example, is noticing how \(\rho_T\) relaxes to the uniform distribution as \(e^{-\sigma^2}\), where, in this scenario, we have
\begin{equation}
	\sigma^2 \propto t^{3}
\end{equation}
while, instead, for a Wiener process, i.e.\ the one that affects \(y\), we expect to have
\begin{equation}
	\sigma^2 \propto t
\end{equation}

\subsubsection*{Application to an Hamiltonian system}

Considering the following Hamiltonian
\begin{equation}
	\ham = \ham_0(I) - \epsilon \theta \xi(t)
\end{equation}
we have that
\begin{equation}
	\begin{aligned}
		\dot{I} &= \epsilon\xi \\
		\dot{\theta} &= \omega(I)
	\end{aligned}
\end{equation}
from which follows immediately
\begin{equation}
	I = I_0 + \epsilon w(t)
\end{equation}
executing than a Taylor expansion for \(\dot{\theta}\), we obtain
\begin{equation}
	\dot{\theta} = \omega(I_0) + \omega'(I_0)\epsilon w(t) + \mathcal{O}\epsilon^2)
\end{equation}
which, if integrated, brings
\begin{equation}
	\theta = \theta_0 + \omega(I_0)t + \epsilon \omega'(I_0) w_1(t)
\end{equation}
this solution, as seen at the beginning of the Section, implies a variance
\begin{equation}
	\sigma_{\theta}^2 \propto t^3
\end{equation}
while instead, for the action variable \(I\), we have a variance
\begin{equation}
	\sigma_{I}^2 \propto t
\end{equation}

\subsection{Averaging a 1D stochastic Hamiltonian}
\label{ssc:averaging-1D}

Let us consider a 1D stochastically perturbed Hamiltonian system, described in action-angle coordinates, with the origin in a critical stable point of \(H_0\), in the form
\begin{equation}
	\ham = \ham_0(I) + \epsilon \xi(t) \ham_1(\theta, I).
\end{equation}
We have that the probability density \(\rho(\theta, I, t)\) satisfies the equation
\begin{equation}
	\pdv{\rho}{t}\/ + \Omega \pdv{\rho}{\theta}\/ = \frac{\epsilon^2 \sigma^2}{2}\{\ham_1, \{\ham_1,\rho\}\}
	\label{eq:stoc_ham_complex_starting}
\end{equation}
where \(\{,\}\) are the Poisson brackets and \(\Omega(I) = \partial \ham_0/\partial I\).

From the analysis in the previous Section, we know that a toroidal angle variable relaxes in much faster times compared to the action relaxation scales (\(t_{\theta} \propto \epsilon^{-2/3}\) for the angle variable while for the action we have instead \(t_{I} \propto \epsilon^{-2}\)). Because of that it is possible to simplify this last equation into a Fokker-Planck equation for \(t\gg t_{\theta}\): the distribution for the action variable \(\rho(I,t)\), averaged over the angle variable, satisfies the equation
\begin{equation}
	\pdv{}{t}\/\rho(I,t) = \pdv{}{I}\/D(I)\pdv{}{I}\/\rho(I,t)
\end{equation}
where
\begin{equation}
	D(I) = \frac{\epsilon^2 \sigma^2}{2}\frac{1}{2\pi}\int_0^{2\pi}\left(\pdv{\ham_1}{\theta}\/\right)^2 \,d\theta \equiv \frac{\epsilon^2 \sigma^2}{2} \left\langle \left( \pdv{\ham_1}{\theta}\/  \right)^2 \right\rangle_\theta
\end{equation}

\paragraph{Proof:}
Let us start by considering the stochastic Liouville equation
\begin{equation}
	\pdv{\rho}{t}\/ + \Omega(I) \pdv{\rho}{\theta}\/ + \epsilon\xi(t)\{\rho,\ham_1\} = 0
	\label{eq:dimo_liouville}
\end{equation}
let us then write \(\rho\) in the form \(\rho = \rho_0 + \epsilon\rho_1\), where \(\rho_0\) is the average component and \(\rho_1\) the fluctuating component with zero average. Considering that \(\langle\xi\rangle = 0\), we have that the average value of Eq.~\eqref{eq:dimo_liouville} is
\begin{equation}
	\pdv{\rho_0}{t}\/ + \Omega(I)\pdv{\rho_0}{\theta}\/ + \epsilon^2\{\langle\xi(t)\rho_1\rangle, \ham_1\} = 0
	\label{eq:dimo_media}
\end{equation}
Subtracting now~\eqref{eq:dimo_media} from~\eqref{eq:dimo_liouville}, we obtain the equation
\begin{equation}
	%TODOTODOTODO... MANCA QUALCOSA??? (da correzioni giovannozzi)
	\pdv{\rho_1}{t}\/ + \Omega(I) \pdv{\rho_1}{\theta}\/ = -\xi(t)\{\rho_0, \ham_1\} + O(\epsilon)
\end{equation}
which we want to solve for \(\rho_1\) so that we can substitute it into Eq.~\eqref{eq:dimo_media}.

To do so, let us first execute the following change of variables
\begin{equation}
	\begin{aligned}
		\theta &\to \theta - \Omega \tau \\
		t &\to t - \tau
	\end{aligned}		
\end{equation}
which allows us to write
\begin{equation}
	\dv{}{\tau}\/\rho_1(\theta - \Omega \tau, I, t - \tau) = \xi(t- \tau)\{\rho_0, \ham_1\}(\theta - \Omega \tau, I, t - \tau)
\end{equation}
integrating this last equation from \(\tau = 0\) to \(\tau = t\), we obtain
\begin{equation}
	\rho_1(\theta, I, t) = -\int_0^t\{\rho_0, \ham_1\}(\theta - \Omega \tau, I, t - \tau)\xi(t-\tau)\,d\tau
\end{equation}
where we took advantage over the fact that \(\rho_1(\theta, I, 0)=0\). Multiplying then both members for \(\xi(t)\) and computing the average over all the noise realizations, we have, in the case of a Wiener noise:
\begin{equation}
	\langle \xi(t)\rho_1(\theta, I, t) \rangle = -\frac{1}{2}\sigma^2 \{\rho_0, \ham_1\}(\theta, I, t)
\end{equation}
if we then replace this result into Eq.~\eqref{eq:dimo_media}, we obtain, excluding terms of order \(\epsilon^3\) and higher, the equation for the average density
\begin{equation}
 	\pdv{\rho_0}{t}\/ + \Omega (I) \pdv{\rho_0}{\theta}\/ = \frac{\epsilon^2 \sigma^2}{2}\{\{\rho_0, \ham_0\}, \ham_0\}.
\end{equation}

And, if we have that the relaxation time scale of the angle is faster than the action diffusion time scale, we can say in good approximation that \(\rho=\rho(I,t)\) does not depend on the angle variable \(\theta\). This leads us to the double Poisson bracket expansion
\begin{equation}
	\{\{\rho_0, \ham_0\}, \ham_0\} = \pdv{}{I}\/\left[\left(\pdv{\ham_1}{\theta}\/\right)^2 \pdv{\rho_0}{I}\/\right] - \pdv{}{\theta}\/\left[\pdv{\rho_0}{I}\/\pdv{\ham_1}{\theta}\/\pdv{\ham_1}{I}\/\right]
\end{equation}
by taking then the angular average of this last equation, the second term in the r.h.s\ becomes zero and we found the expression of \(D(I)\) to be integrated over the whole torus.

\subsection{Averaging principle for generic stochastic Hamiltonians}

We still consider an perturbed Hamiltonian system in action-angle coordinates in presence of a weak chaotic dynamics in phase space, but this time we will consider a more generic approach valid for higher dimensions. We will work with the following stochastically-perturbed Hamiltonian system
\begin{equation}
	\ham = \ham_0(I) + \xi(t)\ham_1(\theta, I)
\end{equation}
where \((\theta, I)\) are the multidimensional action-angle variables and the noise realization \(\xi(t)\) depends on the initial conditions of the orbit.

Let us start by introducing the slow variable
\begin{equation}
	\phi = \theta - \Omega(I)t
\end{equation}
where \(\Omega(I) = \frac{\partial \ham_0}{\partial I}(I)\). This leads us to the following new Hamiltonian
\begin{equation}
	\ham = \xi(t)\ham_1(\phi+\Omega(I)t,I)
\end{equation}
via the generating function
\begin{equation}
	F(\theta,I) = \theta I - \ham_0(I)t
\end{equation}
In order to find an approximate solution of the previous dynamics, we consider the evolution of the angle-action variables for a time \(T\gg \lambda\). This leads us to the map
\begin{equation}
	\begin{aligned}
		\Delta\phi_j(T) &= \int_0^T\pdv{\ham_1}{I_j}\/(\phi+\Omega(I)t, I)\xi(t)\,dt -\\
		&\qquad - \int_0^T t\pdv{\ham_1}{\theta_k}\/(\phi + \Omega(I)t,I)\pdv{\Omega_k}{I_j}\xi(t)\,dt\\
		\Delta I_j(T) &= -\int_0^T\pdv{\ham_1}{\theta_j}\/(\phi+\Omega(I)t,I)\xi(t)\,dt
	\end{aligned}
	\label{eq:averaging-generic-hams}
\end{equation}
where \(\Delta \phi_j(T)=\phi(T)-\phi(0)\) and \(\Delta I(T)=I(T)-I(0)\). We want to inspect the second integral in the formula of \(\Delta\phi_j(T)\) by performing the following integration by parts (we are of course truncating the expansion while doing so)
\begin{multline}
	\int_0^T t\pdv{\ham_1}{\theta_k}\/(\phi+\Omega(I)t,I)\pdv{\Omega_k}{I_j}\xi(t)\,dt \simeq \\
	\simeq \pdv{\Omega_k}{I_j}\/\bigg[T\int_0^T\pdv{\ham_1}{\theta_k}\/(\phi+\Omega(I)t,I)\xi(t)\,dt\, - \\
	- \int_0^T\int_0^t\pdv{\ham_1}{\theta_j}\/(\phi+\Omega(I)s, I)\xi(s)\,ds\,dt\bigg]
\end{multline}
where we can combine these two integrals and replace their arguments using the second equation in~\eqref{eq:averaging-generic-hams} and obtain
\begin{equation}
	\int_0^T t\pdv{\ham_1}{\theta_k}\/(\phi+\Omega(I)t,I)\pdv{\Omega_k}{I_j}\xi(t)\,dt \simeq \pdv{\Omega_k}{I_j}\int_0^T[\Delta I_k(T)-\Delta I_k(t)]\,dt
\end{equation}
Therefore, we can say that if the action dynamics can be considered a stationary process, the main contribution to the angular dynamics is given by
\begin{equation}
	\Delta \phi_j \simeq -\pdv{\Omega_k}{I_j}\int_0^T\Delta I_k(t)\,dt
\end{equation}
Therefore if the system is non-degenerate, i.e.\ the matrix \(\partial\Omega_k/\partial I_j\) is not singular, the increment of the angle variables are the integral of the increments of the action and, as we saw in the previous sections, we expect a much faster relaxation to a uniform distribution. From this result we have that in the evolution of the action variables we can approximate the distribution of the angle variables with a uniform distribution. To get an approximation of the action dynamics, we consider the change of the action up to terms of order \(\mathcal{O}(\norm{\ham_1}^2)\)
\begin{multline}
	\Delta I_j = \int_0^T \pdv{\ham_1}{\theta_j}\xi(s)\,ds + \int_0^T\int_0^t \pdv{\ham_1}{I_k}{\theta_j}\pdv{\ham_1}{\theta_k}\xi(t)\xi(s)\,ds\,dt - \\
	- \int_0^T\int_0^t\pdv{\ham_1}{\theta_k}{\theta_j}\left[\pdv{\ham_1}{I_k}-\pdv{\Omega}{I_k}\pdv{\ham_1}{\theta_k}t\right]\xi(t)\xi(s)\,ds\,dt
\end{multline}
Let us then assume that the angles \(\phi\) are uniformly distributed so that we can average on the noise realizations and on the angles in a single step. The actions \(I\) perform a stochastic dynamics, whose average value is
\begin{equation}
	\begin{aligned}
		\langle\Delta I_j\rangle_\phi &= \pdv{}{I_k}\int_0^T\int_0^t \left\langle\pdv{\ham_1}{\theta_j}\pdv{\ham_1}{\theta_k}\right\rangle e^{-(t-s)/\lambda}\,ds\,dt\\
		&= \frac{1}{2}\pdv{}{I_k}\left\langle\pdv{\ham_1}{\theta_j}\pdv{\ham_1}{\theta_k}\right\rangle\int_0^T\int_0^T e^{-\abs{t-s}/\lambda}\,ds\,dt
	\end{aligned}
\end{equation}
where we have neglected the terms that are derivatves with respect to an angle variable and used the decorrelation law of \(\xi(t)\). The corresponding variance is estimated by % (TODO:UNDERSTAND THIS PASSAGE!!!)
\begin{multline}
	\left\langle(\Delta I_j - \langle\Delta I_j\rangle_\phi)(\Delta I_k - \langle\Delta I_k\rangle_\phi)\right\rangle_\phi =\\
	= \left\langle\pdv{\ham_1}{\theta_j}\pdv{\ham_1}{\theta_k}\right\rangle\int_0^T\int_0^T e^{-\abs{t-s}/\lambda}\,ds\,dt
\end{multline}
In the limit of slow diffusion time and in the approximation of a fast angle relaxation, we can describe the action dynamics by a stochastic process of the form
\begin{equation}
	\Delta I_j = -\sqrt{T\lambda\norm{\ham_1}^2}\sqrt{\left\langle\pdv{\ham_1}{\theta_j}\pdv{\ham_1}{\theta_l}\right\rangle}\,\hat{\xi}_l + \frac{T\lambda\norm{\ham_1}^2}{2}\pdv{}{I_k}\left\langle\pdv{\ham_1}{\theta_j}\pdv{\ham_1}{\theta_k}\right\rangle
	\label{eq:generic-ham-average-almost}
\end{equation}
where \(\hat{\ham_1}=\ham_1/\norm{\ham_1}\) and \(\hat{\xi}_l\) are identical independent distributed random variables with zero mean value and unitary variance. We can interpret the quantity \(T\lambda\norm{\ham_1}^2\) as the time step \(\Delta\tau\) of the diffusion time (even though it has a dimension of an action): the continuous limit is recovered when \(T\to\infty\) (so that \(T\gg\lambda\)) and \(\norm{\ham_1}^2T\to 0\). In other words, \(T\) should be sufficiently long in order to consider the angles relaxing to a uniform distribution and the noise decorrelated, but \(\norm{\ham_1}^2\) has to be so small that the actions do not evolve in a time \(T\). We also recall the limit requested for the validity of the stochastic Langevine equation
\begin{equation}
	\lim_{T\to\infty}\frac{1}{T\lambda}\int_0^T\int_0^T e^{-\abs{t-s}/\lambda}\,ds\,dt = 1
\end{equation}
Finally, introducing the diffusion time notation \(\tau = \lambda\norm{\ham_1}^2 t\), we have that Eq.~\eqref{eq:generic-ham-average-almost} is the approximation of the solution of the stochastic differential equation
\begin{equation}
	dI_j = -\sqrt{\left\langle\pdv{\ham_1}{\theta_j}\pdv{\ham_1}{\theta_l}\right\rangle}\,d\omega_l(\tau) + \frac{1}{2}\pdv{}{I_k}\left\langle\pdv{\ham_1}{\theta_j}\pdv{\ham_1}{\theta_k}\right\rangle\,d\tau
	\label{eq:final-ham-approximation}
\end{equation}

In order to completely check the consistency of our claims, let us consider again the angle dynamics in the diffusion time (we omit for convenience the indices)
\begin{equation}
	\Delta \phi = \left\langle\pdv{\hat{\ham}_1}{I}\right\rangle_{\phi}\sqrt{T\lambda\norm{\ham_1}^2}\,\xi -\pdv{\Omega}{I}\int_0^T(I(T)-I(t))\,dt
\end{equation}
by applying the approximation~\eqref{eq:final-ham-approximation} we just obtained, we can compute directly the fluctuating part of the action term \((I(T)-I(t))\) and obtain
\begin{equation}
	\operatorname{Var}[I(T)-I(t)]=\left\langle\left(\pdv{\hat{\ham}_1}{\theta}\right)^2\right\rangle(\lambda\norm{\ham_1}^2)\frac{T^3}{2}
\end{equation}
At this point, if \(\partial\Omega/\partial I\propto \mathcal{O}(1)\) the stochastic approximation of the action variables we performed implies that the relaxation process for the angles \(\phi\) must be archived after a time \(t_\phi\), where \(\lambda\norm{\ham_1}^2 t_\phi^3 \simeq \mathcal{O}(1)\). Then, we can estimate
\begin{equation}
	t_\phi \simeq \lambda^{-1/3}\norm{\ham_1}^{-2/3}
\end{equation}
so that the choice \(T\simeq t_\phi\) provides the result
\begin{equation}
	\Delta \tau \simeq \lambda^{2/3}\norm{\ham_1}^{4/3}
\end{equation}
which vastly proves the assumption of fast angle relaxation, when compared to the diffusion time scale.

The assumption on the fast relaxation of the angles \(\phi\), necessary to derive the equation~\eqref{eq:final-ham-approximation}, can be satisfied if the estimate \(\norm{\ham_1}^2 \ll \lambda^{-1}\) holds for the decorrelation time scale of the random fluctuations. This condition is necessary to describe the stochastic Hamiltonian dynamics as a diffusion process and it implies that the approach is consistent even if the Ljapounov exponent, characterizing the chaotic region, is small.

The evolution of the distribution function \(\rho(I,\tau)\) at the diffusion time scale is well approximated by the solution of the Fokker-Planck equation
\begin{equation}
	\pdv{\rho}{\tau}=\frac{1}{2}\pdv{}{I_j}\left\langle\pdv{\ham_1}{\theta_j}\pdv{\ham_1}{\theta_k}\right\rangle\pdv{}{I_k}\rho(I,\tau)
\end{equation}
where in this equation the slow diffusion time coefficient \(\tau\) has the dimension of the square of an action, so that the diffusion coefficient is adimensional.


\section{Functional form for $D(I)$}


\section{State of the art in diffusion measurements}