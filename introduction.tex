


\chapter*{Introduction}
\addcontentsline{toc}{chapter}{Introduction}  
\chaptermark{Introduction}

The development and use of particle accelerators have been of paramount importance in making significant discoveries in fundamental particle physics. These large-scale experiments, such as the Large Hadron Collider~\cite{Benedikt:823808} at CERN, require extensive collaboration and expertise across various fields of physics and engineering. The success of these experiments depends heavily on delivering high-quality beams to collision points. This is where the field of accelerator physics comes into play, with a focus on understanding and improving the many aspects of particles' motion inside the accelerator, overcoming the multiple challenges that arise from the complex nature of a high-energy particle beam and the accelerator's environment. The complexity of these challenges increases along with the requirements and targets for future accelerator machines, such as the expected LHC upgrade, HL-LHC~\cite{BejarAlonso:2749422, Arduini_2016}.

The physical concepts driving the dynamics of particles within accelerators are relatively simple. The motion of particles within an accelerator can be reduced to that of a charged, relativistic particle under the influence of an electromagnetic field. As such, the dynamics of particles within accelerators falls under the realm of classical mechanics and can be understood through the use of Hamiltonian formulations of dynamics and, in the context of single-particle tracking, through the use of symplectic maps.

However, in classical mechanics, when we consider the introduction of nonlinear effects and concepts such as chaos, resonances, and integrability, we end up with extremely complex systems, with numerous open problems especially when it comes to determining the long-term evolution of these nonlinear Hamiltonian systems. Research in these areas, which lie at the intersection of mathematics and physics, is ongoing and constantly evolving.

For modern circular particle accelerators, one open problem is the understanding of the complex dynamics of beam-halo formation and evolution, which is essential for optimal design and operation. This is particularly important for colliders utilizing superconducting magnets, such as the LHC, its upgrade HL-LHC, and the proposed FCC-hh~\cite{Benedikt:2651300}, as beam losses have a direct impact on accelerator performance.

The dynamics of the beam-halo is affected by a variety of factors, including nonlinear field errors in superconducting magnets and ripples in their power converters. This in general can be precisely described using a Hamiltonian approach, from which equations of motion can be derived. However, when time-dependent effects are present, it leads to a significant change in the nature of beam dynamics. For example, modulations in the characteristic frequencies of the Hamiltonian system can result in extended weakly-chaotic layers in phase space~\cite{NEISHTADT1991}. In these regions, the orbit diffusion can be modelled as a stochastic process. The complexity increases when the periodic modulations themselves appear to be stochastic in nature, as it could involve the entire accessible phase space.

A recent framework has been proposed~\cite{Bazzani:2019lse,bazzani2020diffusion}, which describes the long-term behaviour of beam dynamics and particle losses in circular accelerators using a diffusive model. This framework uses a Fokker-Planck equation to describe the evolution of the beam distribution and the diffusion coefficient is a key quantity for describing the beam dynamics. The use of diffusive models for transverse dynamics of charged particles in accelerator physics is not new, with a significant amount of literature available on the subject (see for example Refs.~{\cite{Burnod:205343,Meddahi:223301,PhysRevLett.68.33,gerasimov1992applicability,MESS1994279,zimmermann1994transverse,PhysRevLett.77.1051,PhysRevSTAB.5.074001,flilleriii:pac03-rpag004,stancari2011diffusion,stancari:ipac11-tupz033,PhysRevSTAB.15.101001,Stancari:1637929}} and references therein). However, the model developed in this framework has a unique feature, which is the assumption that the functional form of the diffusion coefficient is derived from the optimal estimate of the perturbation series provided by the Nekhoroshev's theorem~\cite{Nekhoroshev:1977aa,Bazzani:1990aa,Turchetti:1990aa}.

The merit of having such global diffusive framework is the possibility of extrapolating the long-term losses, the emittance evolution, and the beam-tail population evolution by using indirect measurements such as the beam loss signal. An increased understanding of each of these quantities would significantly help the improvement of present and future accelerator designs. Moreover, the knowledge of valid scale-laws for long-term extrapolation can also provide a great advantage in the context of simulations and particle tracking, as realistic particle tracking simulations over the order of magnitude of a few minutes of real beam time are currently not feasible.

In this thesis, we begin by providing a complete review of the diffusive framework. Then, the first original contribution will be the implementation of an original measurement protocol, which is optimized for probing a Nekhoroshev-like diffusive coefficient from beam loss data. This protocol makes use of collimator scans, a technique that has been used in the LHC to both inspect the beam tail population and to measure a diffusion coefficient, although a different diffusive framework was used for the latter.

This optimized protocol is then applied to available LHC collimator scan data, which was gathered during a dedicated LHC Run~2 collimator scan measurement campaign. The results of this analysis are presented and discussed in detail, as the data was taken with the aim of measuring a local diffusion coefficient, without making assumptions on the functional form of the global diffusion coefficient.

Another original application of the diffusive framework is then presented in the analysis of the loss signal measured during the testing and application of beam-beam wire compensators on the LHC Beam~2 during a dedicated Run~2 measurement campaign. Wire compensators are a novel device aimed to counter the beam-beam effects in the LHC, which are responsible for beam losses and beam-quality degradation. The analysis of the loss signal via our diffusive model aims to provide insights on the long-term effects of wire compensators on both beam losses and emittance. 

Finally, we also investigate the use of dynamic indicators as a tool for studying the phase space properties of realistic accelerator lattices in single-particle tracking simulations. Dynamic indicators are mathematical tools that inspect the linear response of a system to small perturbations in either its initial conditions or along the orbit, and are used to probe the chaotic character of initial conditions and inspect the geometry of regular orbits. 

Many typologies of dynamic indicators exist, and are widely used mainly in celestial mechanics, which shares some of its Hamiltonian formulations with accelerator physics. Some of these indicators are used also in accelerator physics, such as the Fast Lyapunov Indicator~\cite{Froeschle1997, SZEZECH2005394}, and the Frequency Map Analysis~\cite{Laskar1999,Laskar2003}, while more novel once, such as the Reversibility Error Method~\cite{Panichi2016,Panichi2017} and the Generalized Alignment Index~\cite{Bountis2007,Skokos2015} are still not widely used in the field.

With the intent of also providing new valid tools to the accelerator physics community, we first inspect the classification performances of less known indicators in detecting the chaotic character of initial conditions on a simple accelerator-like nonlinear model, the modulated Hénon map. Next, we apply the achieved knowledge to study the phase space properties of realistic accelerator lattices, with the main goal of identifying a connection between the presence of chaotic regions in the phase space and Nekhoroshev-like diffusive behaviour.

\section*{Structure of the work}

The thesis is divided into three parts. In Part~I, we provide a complete review of the diffusive framework, as well as the fundamental elements of classical mechanics and accelerator physics that are required to follow the rest of this thesis: in Chapter~\ref{ch:mathematical_elements}, we recall the fundamentals of Hamiltonian mechanics along with the notions necessary to understand the Nekhoroshev theorem; in Chapter~\ref{ch:the_diffusive_framework}, we present the elements of stochastic perturbed Hamiltonian systems which, ultimately, lead to the formulation of the diffusive model under study; finally, in Chapter~\ref{ch:accelerator_physics_fundamentals}, we focus on the elements of accelerator physics that are required to understand the context of this thesis.

In Part~II, we begin with the original contributions of this thesis: in Chapter~\ref{ch:probing}, we present the implementation of an optimized measurement protocol for probing a Nekhoroshev-like diffusive coefficient from beam loss data via collimator scans; in Chapter~\ref{ch:diffusion_meas}, we apply this protocol on available LHC collimator scan data; finally, in Chapter~\ref{ch:wire-compensators}, we present the analysis of the loss signal measured during the testing and application of beam-beam wire compensators on the LHC Beam~2.

In Part~III, we focus on the field of single-particle tracking simulations and the usage on dynamic indicators not well known in accelerator physics: in Chapter~\ref{ch:overview_of_dynamic_indicators}, we present a comparative study built on the Hénon map, in which we determine the best performant indicators for an accelerator-like model with time-dependent modulation. Finally, in Chapter~\ref{ch:dyn-lhc}, the results of the study is applied to an HL-LHC lattice, and we investigate its phase space properties by means of selected dynamic indicators.


%%%%%%%%%%%%%%%%%%%%%%%%%%%%%%
% FROM FIRST DIFFUSION ARTICLE %
%%%%%%%%%%%%%%%%%%%%%%%%%%%%%%%%

% For the design and operation of modern circular particle accelerators, understanding the complex dynamics that characterizes the beam-halo formation and evolution is of paramount importance. Indeed, several phenomena leading to particle loss and beam-quality degradation, crucial to determine the performance of a particle accelerator, are closely linked to the evolution of the beam halo. 

% This is particularly true for present and future colliders based on superconducting magnets, such as LHC~\cite{LHCDR}, its upgrade HL--LHC~\cite{BejarAlonso:2749422}, or the proposed FCC-hh~\cite{FCC-hhCDR}. Beam losses have a direct impact on the accelerator performance. 

% Beam-halo dynamics is governed by a multitude of effects, such as the unavoidable nonlinear field errors of the superconducting magnets, as well as ripples in the magnets' power converters. In general, the beam dynamics of hadron machines is accurately described in terms of a Hamiltonian from which the equations of motion can be derived. If the system under consideration includes time-dependent effects, this turns into a radical change of the character of the beam dynamics. For instance, the presence of modulation of the characteristic frequencies of the Hamiltonian system implies the existence of extended weakly-chaotic layers in phase space~\cite{NEISHTADT1991}. In these regions, it is possible to model the orbit diffusion by a stochastic process. The situation worsens in case the periodic modulations themselves resemble stochastic processes since the diffusive behaviour might involve the whole of the accessible phase space.

% Recently, a framework has been developed and proposed~\cite{Bazzani:2019lse,bazzani2020diffusion}, in which the long-term behaviour of the beam dynamics and particle losses in circular accelerators is described by means of a diffusive model. In this framework, the evolution of the beam distribution can be described by a Fokker-Planck (FP) equation, in which the diffusion coefficient represents the key quantity to describe the beam dynamics. The development of diffusive models of the transverse dynamics of charged particles is not at all new for accelerator physics, and a rather broad literature exists (see, e.g.\ Refs.~{\cite{Burnod:205343,Meddahi:223301,PhysRevLett.68.33,gerasimov1992applicability,MESS1994279,zimmermann1994transverse,PhysRevLett.77.1051,PhysRevSTAB.5.074001,flilleriii:pac03-rpag004,stancari2011diffusion,stancari:ipac11-tupz033,PhysRevSTAB.15.101001,Stancari:1637929}} and references therein). However, the model that we developed has a very peculiar feature, since we assume that the functional form of the diffusion coefficient is derived from the optimal estimate of the perturbation series provided by the Nekhoroshev's theorem~\cite{Nekhoroshev:1977aa,Bazzani:1990aa,Turchetti:1990aa}. 

% The FP equation is suitable to study the evolution of a beam distribution in the presence of collimators, whose jaws can be represented by the absorbing boundary conditions needed to solve the FP equation. Furthermore, so-called collimator scans can be used to probe the beam-halo dynamics and, in particular, to reconstruct the behaviour of the diffusion coefficient as a function of transverse amplitude~\cite{MESS1994279,stancari2011diffusion,PhysRevSTAB.16.021003,PhysRevAccelBeams.23.044802}. The method of collimator scans has been intensively used at the LHC: it is based on small displacements of the jaws combined with the measurement of the beam losses. The displacements can be either inward or outward, and depending on the direction, the local losses feature different behaviour. The interpretation of the experimental data relies on a number of assumptions that are closely linked to the form of the FP equation that is used to model the beam dynamics. 

% {A first attempt to validate the proposed model of the diffusion coefficient by means of experimental data has been carried out by re-analysing data collected during past collimator scans at the LHC and is presented in Ref.~\cite{montanari:ipac22-mopost043}. An encouraging agreement, corresponding to an error on the model parameters of the order of $10-15\%$, has been found between the FP model predictions and the experimental data. However, some limitations have been observed that are linked to the protocol used to perform the collimator scans. Indeed, the strongly nonlinear form of the diffusion equation implies that particular care should be taken in the way the amplitude-dependence of the diffusion coefficient is probed experimentally. Furthermore, the experimental protocol should be as much as possible independent from the knowledge of the detail of the transverse beam distribution, as the accurate measurement of the beam tails might be challenging.} In this paper, the properties of the FP equation, in particular that of the outgoing current at a boundary condition, are studied in detail by means of analytical models and even more by means of numerical simulations. These analyses lead to the definition of an optimal protocol to extract the information about the diffusion coefficient by performing a sequence of well-chosen variations of the position of the boundary condition. An important part of our {analyses focuses} on the determination of the accuracy and robustness of the proposed protocol, {which are key aspects for an experimental determination of the form of the diffusion coefficient}.

%%%%%%%%%%%%%%%%%%%%%%%%%%%%%%%%%%
% FROM IPAC PAPER                %
%%%%%%%%%%%%%%%%%%%%%%%%%%%%%%%%%%
% n high-energy colliders or storage rings bound to use superconducting magnets, the beam dynamics is extremely complex and intrinsically nonlinear, due to the unavoidable magnetic field errors. This might generate beam losses or emittance growth that affect the accelerator performance, either because of a reduction of the luminosity or due to a reduction of the operational efficiency. A link between dynamic aperture (DA), i.e. the extent of the phase-space region in which bounded motion occurs, and beam lifetime has been established~\cite{PhysRevSTAB.15.024001} and successfully used to measure DA~\cite{PhysRevAccelBeams.22.034002}. However, this approach does not give any hint on the evolution of the beam distribution, which provides means to predict the beam losses and lifetime, and, more importantly, also the evolution of the beam emittance. This is crucial to assess the presence of emittance growth phenomena, which play a role in determining the actual performance of the collider or storage ring. 

% In this respect, the development of a framework based on diffusive models of the nonlinear beam dynamics is particularly useful. The approach followed is to construct a Fokker-Planck (FP) equation that gives access to the evolution of the beam distribution over time scales compatible with those of physical interest (direct tracking of $10^8$ turns for several initial conditions for a complex lattice like the LHC one is still not an option nowadays). The development of diffusive models of the transverse beam dynamics is not new for accelerator physics (see, e.g.~\cite{PhysRevLett.68.33,gerasimov1992applicability,MESS1994279,zimmermann1994transverse,PhysRevLett.77.1051,PhysRevSTAB.5.074001,stancari2011diffusion,PhysRevSTAB.15.101001} and references therein). However, recently a new framework has been developed~\cite{Bazzani:2019lse,bazzani2020diffusion,our_paper9}, in which the functional form of the diffusion coefficient is derived from the optimal estimate of the perturbation series provided by the Nekhoroshev theorem~\cite{Nekhoroshev:1977aa,Bazzani:1990aa,Turchetti:1990aa}. 
% The FP equation describes the time evolution of the beam distribution $\rho$ according to 
% %
% \begin{equation}
%     \begin{split}
%         \frac{\partial \rho (I, t)}{\partial t} & =\frac{1}{2} \frac{\partial}{\partial I} {D}(I) \frac{\partial}{\partial I}\, \rho(I, t) \\
%         D(I) & \propto\exp \left[-2\left({I_\ast} / {I}\right)^{\frac{1}{2\kappa}}\right]\, , 
%     \end{split}
%     \label{eq:fp}
% \end{equation}
% %
% where $D(I)$ is the diffusion coefficient as a function of the action variable $I$. % has been used. 
% Equation~(\ref{eq:fp}) is suitable for studying the evolution of beam distributions in the presence of collimators with jaws that provide %can be represented by 
% absorbing boundary conditions necessary to solve the FP equation. Note that the so-called collimator scans, i.e. the controlled movement of the jaws of the LHC collimators, can be used to study beam-halo dynamics and, in particular, to reconstruct the diffusion coefficient behaviour as a function of transverse amplitude~\cite{MESS1994279,stancari2011diffusion,PhysRevSTAB.16.021003,PhysRevAccelBeams.23.044802}. The collimator scan method is widely used in LHC operation and is based on small jaw displacements at different amplitudes $I$, combined with measurement of beam losses. Displacements can be inward or outward, causing different characteristic profiles of beam losses. 

% The special functional form of the diffusion coefficient is related to the functional form provided by the estimate of the optimal perturbation series, according to the Nekhoroshev theorem~\cite{Turchetti:1990aa,Bazzani:1990aa}. The parameters $\kappa, I_\ast$ have a physical meaning that stems from the Nekhoroshev theorem: the exponent $\kappa$ is related to the analytic structure of the perturbative series and to the dimensionality of the system~\cite{bazzani2020diffusion}; $I_\ast$ is related to the asymptotic character of the perturbative series.



