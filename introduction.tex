


\chapter*{Introduction}
\addcontentsline{toc}{chapter}{Introduction}  
\chaptermark{Introduction}

%%%%%%%%%%%%%%%%%%%%%%%%%%%%%%%%
% FROM FIRST DIFFUSION ARTICLE %
%%%%%%%%%%%%%%%%%%%%%%%%%%%%%%%%

% For the design and operation of modern circular particle accelerators, understanding the complex dynamics that characterises the beam-halo formation and evolution is of paramount importance. Indeed, several phenomena leading to particle loss and beam-quality degradation, crucial to determine the performance of a particle accelerator, are closely linked to the evolution of the beam halo. 

% This is particularly true for present and future colliders based on superconducting magnets, such as LHC~\cite{LHCDR}, its upgrade HL--LHC~\cite{BejarAlonso:2749422}, or the proposed FCC-hh~\cite{FCC-hhCDR}. Beam losses have a direct impact on the accelerator performance. 

% Beam-halo dynamics is governed by a multitude of effects, such as the unavoidable non-linear field errors of the superconducting magnets, as well as ripples in the magnets' power converters. In general, the beam dynamics of hadron machines is accurately described in terms of a Hamiltonian from which the equations of motion can be derived. If the system under consideration includes time-dependent effects, this turns into a radical change of the character of the beam dynamics. For instance, the presence of modulation of the characteristic frequencies of the Hamiltonian system implies the existence of extended weakly-chaotic layers in phase space~\cite{NEISHTADT1991}. In these regions, it is possible to model the orbit diffusion by a stochastic process. The situation worsens in case the periodic modulations themselves resemble stochastic processes since the diffusive behaviour might involve the whole of the accessible phase space.

% Recently, a framework has been developed and proposed~\cite{Bazzani:2019lse,bazzani2020diffusion}, in which the long-term behaviour of the beam dynamics and particle losses in circular accelerators is described by means of a diffusive model. In this framework, the evolution of the beam distribution can be described by a Fokker-Planck (FP) equation, in which the diffusion coefficient represents the key quantity to describe the beam dynamics. The development of diffusive models of the transverse dynamics of charged particles is not at all new for accelerator physics, and a rather broad literature exists (see, e.g.\ Refs.~{\cite{Burnod:205343,Meddahi:223301,PhysRevLett.68.33,gerasimov1992applicability,MESS1994279,zimmermann1994transverse,PhysRevLett.77.1051,PhysRevSTAB.5.074001,flilleriii:pac03-rpag004,stancari2011diffusion,stancari:ipac11-tupz033,PhysRevSTAB.15.101001,Stancari:1637929}} and references therein). However, the model that we developed has a very peculiar feature, since we assume that the functional form of the diffusion coefficient is derived from the optimal estimate of the perturbation series provided by the Nekhoroshev's theorem~\cite{Nekhoroshev:1977aa,Bazzani:1990aa,Turchetti:1990aa}. 

% The FP equation is suitable to study the evolution of a beam distribution in the presence of collimators, whose jaws can be represented by the absorbing boundary conditions needed to solve the FP equation. Furthermore, so-called collimator scans can be used to probe the beam-halo dynamics and, in particular, to reconstruct the behaviour of the diffusion coefficient as a function of transverse amplitude~\cite{MESS1994279,stancari2011diffusion,PhysRevSTAB.16.021003,PhysRevAccelBeams.23.044802}. The method of collimator scans has been intensively used at the LHC: it is based on small displacements of the jaws combined with the measurement of the beam losses. The displacements can be either inward or outward, and depending on the direction, the local losses feature different behaviour. The interpretation of the experimental data relies on a number of assumptions that are closely linked to the form of the FP equation that is used to model the beam dynamics. 

% {A first attempt to validate the proposed model of the diffusion coefficient by means of experimental data has been carried out by re-analysing data collected during past collimator scans at the LHC and is presented in Ref.~\cite{montanari:ipac22-mopost043}. An encouraging agreement, corresponding to an error on the model parameters of the order of $10-15\%$, has been found between the FP model predictions and the experimental data. However, some limitations have been observed that are linked to the protocol used to perform the collimator scans. Indeed, the strongly nonlinear form of the diffusion equation implies that particular care should be taken in the way the amplitude-dependence of the diffusion coefficient is probed experimentally. Furthermore, the experimental protocol should be as much as possible independent from the knowledge of the detail of the transverse beam distribution, as the accurate measurement of the beam tails might be challenging.} In this paper, the properties of the FP equation, in particular that of the outgoing current at a boundary condition, are studied in detail by means of analytical models and even more by means of numerical simulations. These analyses lead to the definition of an optimal protocol to extract the information about the diffusion coefficient by performing a sequence of well-chosen variations of the position of the boundary condition. An important part of our {analyses focuses} on the determination of the accuracy and robustness of the proposed protocol, {which are key aspects for an experimental determination of the form of the diffusion coefficient}.

%%%%%%%%%%%%%%%%%%%%%%%%%%%%%%%%%%
% FROM IPAC PAPER                %
%%%%%%%%%%%%%%%%%%%%%%%%%%%%%%%%%%
% n high-energy colliders or storage rings bound to use superconducting magnets, the beam dynamics is extremely complex and intrinsically nonlinear, due to the unavoidable magnetic field errors. This might generate beam losses or emittance growth that affect the accelerator performance, either because of a reduction of the luminosity or due to a reduction of the operational efficiency. A link between dynamic aperture (DA), i.e. the extent of the phase-space region in which bounded motion occurs, and beam lifetime has been established~\cite{PhysRevSTAB.15.024001} and successfully used to measure DA~\cite{PhysRevAccelBeams.22.034002}. However, this approach does not give any hint on the evolution of the beam distribution, which provides means to predict the beam losses and lifetime, and, more importantly, also the evolution of the beam emittance. This is crucial to assess the presence of emittance growth phenomena, which play a role in determining the actual performance of the collider or storage ring. 

% In this respect, the development of a framework based on diffusive models of the non-linear beam dynamics is particularly useful. The approach followed is to construct a Fokker-Planck (FP) equation that gives access to the evolution of the beam distribution over time scales compatible with those of physical interest (direct tracking of $10^8$ turns for several initial conditions for a complex lattice like the LHC one is still not an option nowadays). The development of diffusive models of the transverse beam dynamics is not new for accelerator physics (see, e.g.~\cite{PhysRevLett.68.33,gerasimov1992applicability,MESS1994279,zimmermann1994transverse,PhysRevLett.77.1051,PhysRevSTAB.5.074001,stancari2011diffusion,PhysRevSTAB.15.101001} and references therein). However, recently a new framework has been developed~\cite{Bazzani:2019lse,bazzani2020diffusion,our_paper9}, in which the functional form of the diffusion coefficient is derived from the optimal estimate of the perturbation series provided by the Nekhoroshev theorem~\cite{Nekhoroshev:1977aa,Bazzani:1990aa,Turchetti:1990aa}. 
% The FP equation describes the time evolution of the beam distribution $\rho$ according to 
% %
% \begin{equation}
%     \begin{split}
%         \frac{\partial \rho (I, t)}{\partial t} & =\frac{1}{2} \frac{\partial}{\partial I} {D}(I) \frac{\partial}{\partial I}\, \rho(I, t) \\
%         D(I) & \propto\exp \left[-2\left({I_\ast} / {I}\right)^{\frac{1}{2\kappa}}\right]\, , 
%     \end{split}
%     \label{eq:fp}
% \end{equation}
% %
% where $D(I)$ is the diffusion coefficient as a function of the action variable $I$. % has been used. 
% Equation~(\ref{eq:fp}) is suitable for studying the evolution of beam distributions in the presence of collimators with jaws that provide %can be represented by 
% absorbing boundary conditions necessary to solve the FP equation. Note that the so-called collimator scans, i.e. the controlled movement of the jaws of the LHC collimators, can be used to study beam-halo dynamics and, in particular, to reconstruct the diffusion coefficient behaviour as a function of transverse amplitude~\cite{MESS1994279,stancari2011diffusion,PhysRevSTAB.16.021003,PhysRevAccelBeams.23.044802}. The collimator scan method is widely used in LHC operation and is based on small jaw displacements at different amplitudes $I$, combined with measurement of beam losses. Displacements can be inward or outward, causing different characteristic profiles of beam losses. 

% The special functional form of the diffusion coefficient is related to the functional form provided by the estimate of the optimal perturbation series, according to the Nekhoroshev theorem~\cite{Turchetti:1990aa,Bazzani:1990aa}. The parameters $\kappa, I_\ast$ have a physical meaning that stems from the Nekhoroshev theorem: the exponent $\kappa$ is related to the analytic structure of the perturbative series and to the dimensionality of the system~\cite{bazzani2020diffusion}; $I_\ast$ is related to the asymptotic character of the perturbative series.

\section*{Structure of the work}


