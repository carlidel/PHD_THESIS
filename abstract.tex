\thispagestyle{empty}
\pdfbookmark{Abstract}{Abstract}
\chapter*{Abstract} 
\markboth{Abstract}{}

Understanding the complex dynamics of beam-halo formation and evolution in circular particle accelerators is crucial for the design of current and future rings, particularly those utilising superconducting magnets such as the CERN Large Hadron Collider (LHC), its luminosity upgrade HL-LHC, and the proposed Future Circular Hadron Collider (FCC-hh).
 A recent diffusive framework, which describes the evolution of the beam distribution by means of a Fokker-Planck equation, with diffusion coefficient derived from the Nekhoroshev theorem, has been proposed to describe the long-term behaviour of beam dynamics and particle losses.
 In this thesis, we discuss the theoretical foundations of this framework, and propose the implementation of an original measurement protocol based on collimator scans in view of measuring the Nekhoroshev-like diffusive coefficient by means of beam loss data.
 The available LHC collimator scan data, unfortunately collected without the proposed measurement protocol, have been successfully analysed using the proposed framework. This approach is also applied to datasets from detailed measurements of the impact on the beam losses of so-called long-range beam-beam compensators also at the LHC. 
 Furthermore, dynamic indicators have been studied as a tool for exploring the phase-space properties of realistic accelerator lattices in single-particle tracking simulations.
 By first examining the classification performance of known and new indicators in detecting the chaotic character of initial conditions for a modulated Hénon map and then applying this knowledge to study the properties of realistic accelerator lattices, we tried to identify a connection between the presence of chaotic regions in the phase space and Nekhoroshev-like diffusive behaviour, providing new tools to the accelerator physics community.

%%%%%%%%%%%%%%%%%%%%%%%%%%%%%%%%%%%%%%%%%%%%%%%%%%%%%%%%%%%

% In circular particle accelerators, one open problem essential for the design of current and future machines is the understanding of the complex dynamics of beam-halo formation and evolution.
% This is particularly important for colliders utilizing superconducting magnets, such as the LHC, its upgrade HL-LHC, and the proposed FCC-hh, as beam losses have a direct impact on accelerator performance.
% A recent framework has been proposed, which describes the long-term behaviour of beam dynamics and particle losses using a diffusive model. This framework uses a Fokker-Planck equation, whose diffusion coefficient is derived from the optimal estimate of the perturbation series provided by the Nekhoroshev. 
% In this thesis, we propose the implementation of an original measurement protocol of a Nekhorohsev-like diffusive coefficient via beam loss data, based on collimator scan movements.
% This optimized protocol is then applied to available LHC collimator scan data. The results of this analysis are presented and discussed in detail, as the data was taken following a different protocol.
% Another original application of the diffusive framework is then presented in the analysis of the loss signal measured during the testing of beam-beam wire compensators on the LHC.
% Finally, we investigate the use of dynamic indicators as a tool for studying the phase space properties of realistic accelerator lattices in single-particle tracking simulations. 
% With the intent of also providing new valid tools to the accelerator physics community, we first inspect the classification performances of less known indicators in detecting the chaotic character of initial conditions on a modulated Hénon map.
% Next, we apply the achieved knowledge to study the properties of realistic accelerator lattices, with the goal of identifying a connection between the presence of chaotic regions in the phase space and Nekhoroshev-like diffusive behaviour.
